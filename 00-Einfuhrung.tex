\chapter{Einführung}

Die Stochastische Analysis verbindet Konzepte der Analysis mit denen der
Wahrscheinlichkeitstheorie. Eine zentrale Frage ist, wie sich die wohlbekannten
Kalküle wie Differentiation und Integration übertragen, wenn die
zugrundeliegenden Objekte auch vom Zufall abhängen.

Sie spielt in vielen Anwendungen eine fundamentale Rolle, z.B. in der
Finanzmathematik, der stochastischen Physik und bei der Modellierung von
biologischen Prozessen.

\section{Eine informelle Einführung}

Einen \emph{stochastischen Prozess} kann man als eine Familie von
Zufallsvariablen auffassen. Im diskreten Fall ist diese Familie über eine
endliche bzw. abzählbare Indexmenge parametrisiert und man erhält Folgen von
Zufallsvariablen. Im kontinuierlichen Fall dagegen ist die Indexmenge
überabzählbar, z.B. ein Intervall, eine offene Teilmenge des $\R^d$ oder eine
Teilmenge eines Hilbertraums.

Als zentrales Beispiel betrachten wir im Folgenden einen Aktienkurs
\begin{align*}
S = (S_t)_{t\in[0,T]},
\end{align*}
dabei ist $S_t$ für jedes $t\in[0,T]$ eine
$\log$-normalverteilte Zufallsvariable. Sind die Pfade $t\mapsto S_t(\omega)$
stetig, kann man $S$ auch als Abbildung vom Ereignisraum in die stetigen
Abbildungen auffassen
\begin{align*}
S : \Omega \to C^0([0,T]),
\end{align*}
oder als reellwertige Abbildung, die stetig von einem Parameter abhängt,
\begin{align*}
S: \Omega\times[0,T]\to \R.
\end{align*}
Die einfachste Interpretation ist jedoch die, dass $S$ jedem Zeitpunkt $t$ eine
Zufallsvariable $S_t$ zuordnet.

Wie entwickelt sich der Aktienkurs? Ein erstes Modell stellte 1900 Louis
Bachelier \footnote{Louis Bachelier (* 11. März 1870 in Le Havre; † 26. April 1946 in
St-Servan-sur-Mer) war ein französischer Mathematiker.} vor. Er vermutete,
dass die Zuwächse $\Delta S_t$ einem deterministischen Trend folgen, welcher
durch eine zufällige Komponente gestört wird, 
\begin{align*}
\Delta S_t = \mu \Delta t + \sigma \Delta W_t.
\end{align*}
Für kleine Zeiträume von etwa einem Tag oder einem Monat ist dieses Modell
durchaus annehmbar.
Die zufällige Komponente $\Delta W_t$ ist hier eine $N(0,\Delta t)$-verteilte
Zufallsvariable und $W_t$ eine Brownsche Bewegung.

\begin{defn}
Eine \emph{Standard-Brownsche Bewegung} (auch \emph{Wiener-Prozess}) ist ein
stochastischer Prozess mit unabhängigen, stationären und
normalverteilten Zuwächsen. Es gilt also $W_{0} = 0$ und für
\begin{align*}
0 = t_0 < t_1 < \ldots < t_n = T
\end{align*}
sind die
\begin{align*}
\Delta W_k = W_{t_{k+1}}-W_{t_{k}}
\end{align*}
unabhängige und $N(0,t_{k+1}-t_k)$-verteilte Zufallsvariablen.\fish
\end{defn}

Meist nehmen wir noch zusätzlich an, dass die Pfade stetig sind,
d.h. $W_t(\omega)$ ist stetig in $t$ für jedes $\omega\in\Omega$.

Formal erhält man beim Übergang $\Delta t \to 0$ ein infinitesimales
Modell für den Aktienkurs
\begin{align*}
\ddd S_t = \mu \dt + \sigma \ddd W_t.\tag{\ensuremath{\star}}
\end{align*}
Hierbei ist jedoch zunächst nicht klar, wie die Differentiale zu interpretieren
sind. In der Tat ist eine Brownsche Bewegung $W_t$ bis auf eine Nullmenge
nirgends differenzierbar, denn ihre \emph{Totalvariation} ist unbeschränkt,
\begin{align*}
\Var\left(\frac{\Delta W_t}{\Delta t} \right) = 
\frac{1}{\Delta t^2}\Var(\Delta W_t) = \frac{1}{\Delta t}\to \infty,\qquad
\Delta t \to 0.
\end{align*}
Eine Berechtigung für die Schreibweise ($\star$) liefert die Integration der
Gleichung,
\begin{align*}
S_t = S_0 + \mu t + W_t.
\end{align*}
Es ist ein fundamentales Prinzip, dass stochastische Differentialgleichungen
stets durch die zugehörigen stochastischen Integralgleichungen interpretiert
werden.

Ein Problem des Modells von Bachelier ist jedoch, dass die Wahrscheinlichkeit
für einen negativen Aktienkurs strikt positiv ist
\begin{align*}
P(S_t < 0)  > 0,
\end{align*}
und dies wiederspricht der Realtiät. Dieser Misstand wurde erst 1965 durch ein
alternatives Modell von Paul Samuelson\footnote{Paul Anthony Samuelson (* 15.
Mai 1915 in Gary, Indiana; † 13. Dezember 2009 in Belmont, Massachusetts[1])
war ein US-amerikanischer Wirtschaftswissenschaftler und Träger des
Wirtschaftsnobelpreises von 1970.} behoben. Er leitete aus der Beobachtung, dass
hohe Kurse auch große Änderungsraten mit sich führen, folgende Beschreibung der
Zuwächse ab
\begin{align*}
\Delta S_t = \mu S_t \Delta t + \sigma S_t \Delta W_t.
\end{align*}
In diesem Modell werden also nicht die Zuwächse sondern die Renditen nach
Bachelier modelliert
\begin{align*}
\frac{\Delta S_t}{S_t} = \mu \Delta t  + \sigma \Delta W_t.
\end{align*} 
Statistische Beobachtungen belegen, dass Samuelsons Modell die Wirklichkeit
wesentlich besser beschreibt als das von Bachelier.

Auch hier erhält man durch den Übergang $\Delta t \to 0$ eine infinitesimale
Version
\begin{align*}
\ddd S_t = \mu S_t \dt + \sigma S_t \ddd W_t,
\end{align*}
wobei bei Betrachtung der Differentiale ähnliche Probleme wie vorhin auftreten.
Nach Integration erhält man die stochastische Integralgleichung
\begin{align*}
S_t - S_0 = \int_0^t \mu\, S_\tau\, \ddd \tau + \int_0^t \sigma\, S_\tau\,
\ddd W_\tau.
\end{align*}

Hier stellt sich nun die Frage wie der Integrand $S_\tau \,\ddd W_\tau$ zu
interpretieren ist, denn aufgrund der Irregularität von $W_t$ ist eine
Auffassung als Riemann-Stieltjes-Integral nicht möglich. 
Im Laufe dieser Vorlesung widmen wir uns zunächst Fragen zur Existenz und
Eindeutigkeit von Lösungen solcher Integralgleichungen. Weiterhin suchen
wir nach Wegen, diese Lösungen zu beschaffen. Wie im klassischen Fall
werden wir feststellen, dass für die Mehrzahl dieser Probleme keine
geschlossenen Ausdrücke existieren, daher beschäftigen wir uns
auch mit numerischer Approximation.

Als Motivation betrachten wir nun ein elementares Integral
\begin{align*}
\int_0^t W_\tau\, \ddd W_\tau.
\end{align*}
Rechnen wir strikt formal nach dem aus der Analysis bekannten Kalkül, erhalten
wir
\begin{align*}
\int_0^t W_\tau\, \ddd W_\tau = \frac{1}{2} \int_0^t \ddd (W_\tau^2) = 
\frac{1}{2}W_\tau^2\bigg|_{0}^t.
\end{align*}
Es stellt sich nun heraus, dass diese Rechnung nicht korrekt ist, denn es gilt
\begin{align*}
\int_0^t W_\tau\, \ddd W_\tau = \frac{1}{2} W_t^2 - \frac{1}{2}t^2,
\end{align*}
zumindest im $L^2$-Sinn. Wir wollen dies nun herleiten. 
Unter der Voraussetzung, dass die Pfade stetig sind, können wir das Integral auch als
Limes von Riemmann-Summen betrachten
\begin{align*}
\int_0^t W_\tau\, \ddd W_\tau = \lim\limits_{n\to \infty} \sum_{i=0}^{n-1}
W_{t_i}(W_{t_{i+1}}-W_{t_i}),
\end{align*}
wobei wir $W_\tau$ am linken Randpunkt $t_i$ auswerten. Diese Wahl
erleichtert nicht nur die Rechnung, sondern bewirkt auch, dass die entstehenden
Objekte Martingale und folglich sehr angenehm zu handhaben sind. Um den
Grenzwert zu berechnen, betrachten wir zunächst
\begin{align*}
\sum_{i=0}^{n-1}
W_{t_i}(W_{t_{i+1}}-W_{t_i}) &=
\frac{1}{2}
\sum_{i=0}^{n-1}
(W_{t_{i+1}}^2-W_{t_i}^2) 
-\frac{1}{2}\sum_{i=0}^{n-1}
(W_{t_{i+1}}-W_{t_i})^2 
\\
&= 
\frac{1}{2}(W_t^2 - W_0^2)-\frac{1}{2}\sum_{i=0}^{n-1}
(W_{t_{i+1}}-W_{t_i})^2.
\end{align*}
Der Erwartungswert des letzten Terms ist
\begin{align*}
\E\left(\sum_{i=0}^{n-1}
(W_{t_{i+1}}-W_{t_i})^2 \right)=
\sum_{i=0}^{n-1} \Var (W_{t_{i+1}}-W_{t_i}) = 
\sum_{i=0}^{n-1} (t_{i+1}-t_i) = t,
\end{align*}
denn die Zuwächse $W_{t_{i+1}}-W_{t_i}$ sind $N(0,t_{i+1}-t_i)$
verteilt. Weiterhin gilt
\begin{align*}
\E \left(\sum_{i=0}^{n-1}
(W_{t_{i+1}}-W_{t_i})^2 - t \right)^2
&= \Var \left(\sum_{i=0}^{n-1}
(W_{t_{i+1}}-W_{t_i})^2 \right)\\
&= 
\sum_{i=0}^{n-1}
\Var (W_{t_{i+1}}-W_{t_i})^2,
\end{align*} 
nach dem Satz von Bienaymé. Nach Voraussetzung ist die Varianz der Zuwächse
$\Delta t$, also $W_{t_{i+1}}-W_{t_i} = \sqrt{\Delta t} \cdot Z$ mit einer 
standard normalverteilten Zufallsvariablen $Z$. Somit gilt
\begin{align*}
\sum_{i=0}^{n-1}
\Var (W_{t_{i+1}}-W_{t_i})^2 = 
 \sum_{i=0}^{n-1} \Delta t^2\Var(Z^2) = 3 t\cdot \Delta t,
\end{align*}
d.h. es liegt Konvergenz im $L^2$-Sinn vor
\begin{align*}
\sum_{i=0}^{n-1} W_{t_i}(W_{t_{i+1}}-W_{t_i})
\quad\overset{L^2}{\longrightarrow}\quad
\frac{1}{2}W_t^2 - \frac{1}{2}W_0^2 - \frac{1}{2}t
\end{align*}
und folgende Definition ist sinnvoll
\begin{align*}
\int_0^t W_\tau\,\ddd W_\tau = \frac{1}{2}W_t^2 - \frac{1}{2}t.
\end{align*}

Wir benötigen also einen neuen Kalkül, um Integrale dieser Art zu berechnen.
Die aus der Analysis bekannten Methoden wie der Hauptsatz, partielle Integration
oder Substitution sind so nicht anwendbar, und der Weg über Riemmann-Summen ist
aufwändig und lästig.

Die von \Ito\footnote{\Ito Kiyoshi (* 7. September 1915 in Hokusei-chō
(heute Inabe), Präfektur Mie; † 10. November 2008 in Kyōto) war ein japanischer
Mathematiker.} entwickelte Theorie verallgemeinert die oben genannten Konzepte
der Analysis auf Brownsche Bewegungen. Ist $f$ eine differenzierbare
Funktion und $g$ von beschränkter Variation, so gilt
\begin{align*}
f(g(t)) = f(g(0)) + \int_0^t f'(g(\tau))\dg(\tau).
\end{align*}
Wählen wir $g(\tau) = W_\tau$, so ist nach obiger Rechnung eine Korrektur dieses
Terms notwendig. Genauer gilt die sogenannte \Ito-Formel
\begin{align*}
f(W_t) = f(W_0) + \int_0^t f'(W_\tau)\,\ddd W_\tau + \frac{1}{2}\int_0^t
f''(W_\tau)\,\dtau.
\end{align*}
Grob gesprochen bezahlt man durch den letzten Term eine Strafe für die
Irregularität der Pfade von $W_t$. Zur Motivation der \Ito-Formel betrachten wir
eine $C^3$-Funktion $f$ und zerlegen $[0,t]$ in äquidistante Teilintervalle. Mit der
Taylorformel erhalten wir dann
\begin{align*}
f(W_t) &= f(W_0) = \sum_{i=0}^{n-1} (f(W_{t_{i+1}}) - f(W_{t_i}))\\
 &= \sum_{i=0}^{n-1} \left(f'(W_{t_i})(W_{t_{i+1}}-W_{t_i}) + 
 \frac{1}{2}f''(W_{t_i})(W_{t_{i+1}}-W_{t_i})^2 + \ldots\right).
\end{align*}
Wäre $W$ von beschränkter Totalvariation, so würden für $\Delta t \to 0$ alle
Terme von 2. Ordnung und höher verschwinden. Dies ist aber nicht der Fall, denn
es gilt lediglich $(W_{t_{i+1}}-W_{t_i})^2 = O(\Delta t)$. Somit verschwinden
nur die Terme der Ordnung 3 und höher, und die \Ito-Formel folgt.

In der etwas allgemeineren Situation von $f\in C^{1,2}$, d.h. $f(t,x)$ ist
$C^1$ in $t$ und $C^2$ in $x$, existiert ebenfalls eine \Ito-Formel, nämlich
\begin{align*}
f(t,W_t) - f(0,W_0) = \int_0^t \dot{f}(\tau,W_\tau)\dtau + \int_0^t
f'(\tau,W_\tau)\ddd W_\tau + \int_0^t f''(\tau,W_\tau)\dtau, 
\end{align*}
oder kurz
\begin{align*}
\df = \dot f \dt + f' \ddd W + \frac{1}{2} f'' \dt.
\end{align*}
Mit dieser Formel können wir zeigen, dass der Prozess
\begin{align*}
S_t = S_0\exp(\sigma W_t + \mu t + \frac{1}{2}\sigma^2
t)\tag{\ensuremath{\star\star}}
\end{align*}
die stochastische Differentialgleichung
\begin{align*}
\ddd S_t = \mu S_t \dt + \sigma S_t \ddd W_t
\end{align*}
des Samuelson Modells für Aktienkurse löst. Setzen wir nämlich $f(t,W_t) \defl
S_t$, dann gilt nach obiger \Ito-Formel
\begin{align*}
\df &= \ddd S_t = \left(\mu-\frac{1}{2}\sigma^2 \right)S_t \dt + \sigma S_t
\ddd W_t + \frac{1}{2}\sigma^2 S_t\ddd W_t\\
& = \mu S_t \dt + \sigma S_t \ddd W_t.
\end{align*} 

Wir werden die \Ito-Formel für Semimartingale beweisen. 
Diese bilden eine große Klasse von
stochastischen Prozessen und schließen auch die Brownsche Bewegung
mit ein. Anschließend werden wir eine Substitutionsformel sowie eine partielle Integration herleiten. Damit erhalten
wir einen Kalkül zur eleganten Analyse von stochastischen Differential- und
Integralgleichungen, mit dessen Hilfe wir die berühmte
Black \& Scholes Formel erarbeiten können.
