\chapter{Stochastische Integration vorhersagbarer Prozesse}

Unser stochastisches Integral haben wir zunächst für einfach vorhersagbare
Integranden definiert, und anschließend unter Verwendung der Stetigkeit des
Integrals auf linksstetige Prozesse fortgesetzt. Technisch entspricht dieses
Integral dem Riemann-Integral, welches ebenfalls nur für Funktionen erklärt
werden kann, deren Unstetigkeitsstellen Maß Null haben.
Wir wollen das stochastische Integral nun auf eine wesentlich größere Klasse
von Integranden ausdehnen, nämlich die vorhersagbaren Prozesse. Technisch
entspricht dieser Schritt der Konstruktion des Lebesgue-Integrals, welches auch
Funktionen mit Unstetigkeitsstellen von positivem Maß zulässt -- beispielsweise
die Dirichletfunktion. Es stellt sich heraus, dass man die Klasse der
Integranden nicht wesentlich über die vorhersagbaren Prozesse hinaus vergrößern
kann, wenn die Integration abgeschlossen unter lokalen Martingalen bzw.
Semimartingalen sein soll.

\section{Integration beschränkter Semimartingale}

Um die Notation etwas zu vereinfachen, wollen wir in Zukunft immer voraussetzen,
dass $X$ ein Semimartingal ist mit $X_0 = 0$. Haben wir einmal das stochastische
Integral für diese Klasse von Integratoren erklärt, so erhalten wir direkt ein
Integral bezüglich beliebiger Semimartingale $X=\hat{X}+X_0$, indem wir
definieren
\begin{align*}
\int_0^t H_s \,\dX_s = \int_0^t H_s\,\ddd\hat{X}_s + H_0 X_0.
\end{align*}

\subsection{Die Topologie der Integratoren}

In Kapitel \ref{sec:stoch-int} haben wir die einfach vorhersagbaren Prozesse
$\S$ mit der $\ucp$-Topologie versehen, und das Integral als stetige Abbildung
\begin{align*}
J_X : \S_{\ucp}\to \D_{\ucp}
\end{align*} 
definiert. Da die einfach vorhersagbaren Prozesse $\S$ in der $\ucp$-Topologie
dicht in den linksstetigen Prozessen $\L$ liegen, ließ sich $J_X$ unter
Verwendung der Stetigkeit von $\S$ auf ganz $\L$ fortsetzen.

Analog wollen wir nun eine geeignete Topologie auf den vorhersagbaren Prozessen
$\Pc$ definieren, so dass einerseits $\L$ bezüglich dieser Topologie dicht in
$\Pc$ liegt, und andererseits das bereits bekannte stochastische Integral
Cauchyfolgen $(H^n)$ dieser Topologie auf Cauchyfolgen $(H^n\bullet X)$
abbildet. Mit Hilfe der Vollständigkeit des zugrunde liegenden Raumes lässt sich
dann ein Integral für allgemeine Integranden aus $\Pc$ als Limes der
Approximationen $H^n\bullet X$ definieren.
 
\nomenclature{$\Pc$}{vorhersagbare $\sigma$-Algebra bzw. vorhersagbare
Prozesse} Erinnern wir uns zunächst an die Definition der Vorhersagbarkeit
\ref{defn:3.2}. Zunächst bezeichnet $\Pc$ die kleinste $\sigma$-Algebra auf
$\R_+\times\Omega$, so dass alle Prozesse aus $\L$ messbar sind.
Interpretieren wir also einen Prozess $X$ als Abbildung
\begin{align*}
X: \R_+\times\Omega\to \R,
\end{align*} 
so können wir die vorhersagbare $\sigma$-Algebra auch darstellen als
\begin{align*}
\Pc = \sigma\left(\bigcup_{X\in\L} \Pc_X\right),\qquad \Pc_X = X^{-1}(\Bc). 
\end{align*}
Die $\Pc$-messbaren Prozesse bezeichnen wir ebenfalls mit $\Pc$, wobei aber
aus dem Zusammenhang stets klar sein sollte, was gemeint ist.

Wir wollen nun eine geeignete Topologie auf $\Pc$ definieren. 
In Kapitel \ref{c:stoch-int} genügte es, für alle möglichen Integratoren
gemeinsam eine einzige Topologie zu betrachten, die \ucp-Topologie. Unabhängig
vom betrachteten Semimartingal $X$ ist die Abbildung
\begin{align*}
J_X : \S \to \D
\end{align*}
bezüglich dieser Topologie stetig. Dies ist nicht mehr möglich, wenn wir das
Integral noch weiter verallgemeinern wollen. Dazu müssen wir die Topologie
einzeln an den jeweiligen Integrator anpassen.

\begin{rem*}[Bemerkung zur Notation.]
\begin{remenum}
\item
Nach Satz \ref{prop:3.4} lässt sich jedes spezielle Semimartingal eindeutig in
ein lokales Martingal und einen vorhersagbaren Prozess von beschränkter Variation zerlegen. Diese kanonische
Zerlegung bezeichnen wir im Folgenden mit
\begin{align*}
X = \bar{N}+\bar{A}.
\end{align*}
\item Das Differential des Variationsprozesses $(\abs{\bar{A}}_t)_{t\ge 0}$
bezeichnen wir mit $\abs{\ddd \bar{A}_s}$, um mit der einschlägigen Literatur
kompatibel zu sein. Gemeint ist damit $\ddd \abs{\bar{A}}_s$.\map
\end{remenum}
\end{rem*}

\begin{definition}
\label{defn:4.1}
\nomenclature[S]{$\Hs^2$}{siehe}
\nomenclature[C]{$X_n\Hto X$}{$X_n\to X$ in $\Hs^2$}
\index{$\Hs^2$-Norm}
Die \emph{$\Hs^2$-Norm} eines speziellen Semimartingals
$X=\bar{N}+\bar{A}$ ist definiert durch
\begin{align*}
\norm{X}_{\Hs^2} \defl \norm*{[\overline{N},\overline{N}]_\infty^{1/2}}_{L^2}
+
\norm*{\int_0^\infty |\ddd\overline{A}_s|}_{L^2},
\end{align*}
falls dieser Ausdruck endlich
ist. Der Raum der speziellen Semimartingale mit endlicher $\Hs^2$-Norm wird
mit \emph{$\Hs^2$} bezeichnet.\fish
\end{definition}
\begin{proof}[Nachweis der Normeigenschaften]
Die speziellen Semimartingale bilden einen linearen Raum, welcher die
Menge $\Hs^2$ enthält. Die Homogenität der Abbildung $\norm{\cdot}_{\Hs^2}$ und
die Dreiecksungleichung sind offensichtlich, also ist $\Hs^2$ sogar ein
linerarer Teilraum, und $\norm{\cdot}_{\Hs^2}$ ist dort eine Halbnorm. Sei
$X=\bar{N}+\bar{A}\in\Hs^2$, dann gilt insbesondere
\begin{align*}
\norm*{[\overline{N},\overline{N}]_\infty^{1/2}}_{L^2}^2 = 
\E [\overline{N},\overline{N}]_\infty < \infty.
\end{align*}
Mit Satz \ref{prop:2.18} folgt, dass $N$ ein $L^2$-Martingal ist mit
$\norm{[\overline{N},\overline{N}]_t^{1/2}}_{L^2}^2 = \E
[\overline{N},\overline{N}]_t = \E N_t^2$ für alle $0\le t \le \infty$.
Sei also $\norm{X}_{\Hs^2} = 0$, dann gilt $\E
N_\infty^2 = 0$, und folglich ist $N_t= 0$ f.s. für alle $t$. Weiterhin ist
$\norm{\int_0^\infty \abs{\dA_s}}_{L^2} = 0$, d.h. $\abs{A}_t = 0$ f.s. für alle
$t \ge 0$ und folglich ist $\bar{A}$ konstant, also $\bar{A} \equiv \bar{A}_0 =
0$.
Somit folgt aus $\norm{X}_{H^2} = 0$ auch $X = 0$.\qed
\end{proof}

Um die Funktionalanalysis effektiv anwenden zu können, benötigen wir die
Vollständigkeit des Raumes $\Hs^2$.

\begin{theorem}
\label{prop:4.1}
Der Raum $\Hs^2$ ist ein Banachraum.\fish
\end{theorem}
\begin{proof}
Aufgrund der Semimartingaleigenschaft zerfällt $\Hs^2$ in zwei Summanden
\begin{align*}
\Hs^2 = \Ms^2  \oplus \Ac,\qquad \Ms^2 \defl \setd{X=\bar{N}\in \Hs^2},\quad
\Ac \defl \setd{X=\bar{A}\in \Hs^2}.
\end{align*}
Die Norm auf $\Hs^2$ entspricht gerade der Norm der direkten Summe, es genügt
daher zu zeigen, dass jeder Summand vollständig ist.

Sei also $\bar{N}\in\Ms^2$, dann ist $\bar{N}$ ein $L^2$-Martingal welches durch
$\bar{N}_t = \E(\bar{N}_\infty\mid\Fc_t)$ mit seinem Abschluss identifiziert
werden kann. Außerdem gilt $\norm{\bar{N}}_{\Hs^2} =
\norm{\bar{N}_\infty}_{L^2}$, also ist die Abbildung
\begin{align*}
\Phi: \Ms^2\to L^2,\qquad \bar{N}\mapsto \bar{N}_\infty,
\end{align*}
ein isometrischer Isomorphismus. Aus der Vollständigkeit von $L^2$ folgt, dass
$\Ms^2$ vollständig ist.

Wir zeigen als nächstes, dass jede absolut konvergente Reihe in $\Ac$
konvergiert. Dann folgt die Vollständigkeit, denn sei eine Cauchyfolge $(A^n)$
in $\Hs^2$ gegeben, so können wir eine Teilfolge $(A^{n_k})$ so wählen, dass
$\norm{A^{n_{k}}-A^{n_{k-1}}}_{\Hs^2} \le 2^{-k}$ ist. Also konvergiert
$\sum_{i\ge 1} \norm{A^{n_{i}}-A^{n_{i-1}}}_{\Hs^2}$ und folglich existiert
existiert der $\Hs^2$-Limes
\begin{align*}
\lim\limits_{k\to \infty} A^{n_k} = A^0 + \sum_{i\ge 1} (A^{n_{i}}-A^{n_{i-1}})
\end{align*}
Also hat die Cauchyfolge $(A^n)$ eine konvergente Teilfolge und ist
daher selbst konvergent.

Sei nun $\sum_{n\ge 1} A^n$ eine absolut konvergente Reihe in $\Ac$. Mit der
$L^2$-Dreiecksun"=gleichung folgt
\begin{align*}
\biggl\|\sum_{n\ge 1} \int_0^\infty \abs{\dA_s^n}\biggr\|_{L^2} \le
\sum_{n\ge 1} \norm*{ \int_0^\infty \abs{\dA_s^n}}_{L^2} = 
\sum_{n\ge 1} \norm{A^n}_{\Hs^2} < \infty,
\end{align*}
also ist $\sum_{n\ge 1} \int_0^\infty \abs{\dA_s^n}$ f.s. endlich. Insbesondere
gilt dann
\begin{align*}
\sum_{n\ge 1}\abs{A_t^n} = \sum_{n\ge 1}\abs{A_t^n-A_0^n} \le
\sum_{n\ge 1}\abs{A^n}_t \le \sum_{n\ge 1}\int_0^\infty \abs{\dA_s^n} <
\infty\quad \fs,
\end{align*}
wobei die Ausnahmemenge nicht von $t$ abhängt. Somit existiert der
Limes punktweise
\begin{align*}
A \defl \sum_{n\ge 1} A^n \quad \fs.
\end{align*}
Alle $A^n$ sind $\Pc$-messbar, also ist es auch $A$, d.h. $A$ ist vorhersagbar.
Außerdem ist $A$ von beschränkter Varition, denn sei $t > 0$ fest und
$\sigma=(T_i^m)$ eine Partition von $[0,t]$, welche gegen die Identität
konvergiert, dann gilt
\begin{align*}
\sum_{i\ge 1} \abs{A_{T_{i}^m}- A_{T_{i-1}^m}}
&= \sum_{i\ge 1} \abs{\sum_{n\ge 1} (A_{T_{i}^m}^n- A_{T_{i-1}^m}^n)}\\
&\le
\sum_{n\ge 1} \sum_{i\ge 1} \abs{A_{T_{i}^m}^n- A_{T_{i-1}^m}^n}\\
&\le \sum_{n\ge 1} \abs{A^n}_t
\le \sum_{n\ge 1} \int_0^\infty \abs{\dA_s^n} < \infty. 
\end{align*}
Somit ist $A\in\Hs^2$. Setzen wir nun $R^N = A-\sum_{1\le n\le N} A^n$, so ist
$R^N\in\Hs^2$, und
\begin{align*}
\sum_{i\ge 1} \abs{R^{T_i^m}-R^{T_{i-1}^m}}
\le
\sum_{n\ge N}\sum_{i\ge 1} \abs{A_{T_i^m}^n-A_{T_{i-1}^m}^n}
\le
\sum_{n\ge N} \int_0^\infty \abs{\dA_s^n}.
\end{align*}
Die rechte Seite bildet eine $L^2$-integrierbare Majorante, also gilt auch
\begin{align*}
\sum_{1\le i\le N}A^n \overset{\Hs^2}{\longto }A,
\end{align*}
und $\Ac$ ist vollständig.\qed
\end{proof}

Im nächsten Schritt wollen wir zeigen, dass die Integration abgeschlossen in
$\Hs^2$ ist, d.h. dass $H\bullet X\in\Hs^2$, wenn $X\in\Hs^2$ und $H\in\bL$.
Dazu benötigen wir folgendes Zwischenergebnis.

\begin{lemma}
\label{lem:4.1}
Sei $A$ ein vorhersagbarer FV Prozess und $H\in\L$ mit
$\E\int_0^\infty \abs{H_s} \,\abs{\dA_s} < \infty$. Dann ist der
stochastische Prozess $\left( \int_0^t H_s\, \dA_s \right)_{t \ge 0}$ FV und
vorhersagbar.\fish
\end{lemma}
\begin{proof}
Sei $(\sigma_n)$ eine Folge zufälliger Partitionen, welche gegen die Identität
konvergiert. So gilt
\begin{align*}
\sum_{i=0}^{k_n-1} H_{T_i^n}(A_t^{T_{i+1}^n}-A_t^{T_i})\ucpto \int H_s\,\dA_s.
\end{align*}
Da $A$ vorhersagbar und $H$ linksstetig ist, ist jedes
$H_{T_i^n}(A_t^{T_{i+1}^n}-A_t^{T_i})$ vorhersagbar, und folglich auch der
stochastische Limes $H\bullet A$.

Weiterhin ist nach Voraussetzung
$\E\int_0^\infty \abs{H_s} \,\abs{\dA_s} < \infty$, also gilt $\int_0^\infty
\abs{H_s} \,\abs{\dA_s} < \infty$ \fs, und daher gilt für jedes $t\ge 0$
\begin{align*}
\Var\left(\int_0^t
H_s \,\dA_s\right) \le \int_0^\infty \abs{H_s}\,\abs{\dA_s} < \infty\quad
\fs.\qed
\end{align*}
\end{proof}

% \begin{cor*}
% $\Hs^2$ ist für Integranden aus $\bL$ abgeschlossen unter stochastischer
% Integration.\fish
% \end{cor*}
% \begin{proof}
Seien nun $X\in\Hs^2$ und $H\in\bL$ gegeben, so ist das stochastische Integral
$H\bullet X$ wohldefiniert, und zerfällt unter der kanonischen Zerlegung von
$X$ in
\begin{align*}
H\bullet X = H\bullet \bar{N} + H\bullet \bar{A}.
\end{align*}
Dabei ist $H\bullet \bar{N}$ wieder ein lokales Martingal, und $H\bullet
\bar{A}$ ist ein vorhersagbarer FV-Prozess. Also ist $H\bullet X$ nach
Definition \ref{defn:3.3} wieder ein spezielles Semimartingal, und die Zerlegung
ist nach Satz \ref{prop:3.4} eindeutig.

Nach Voraussetzung ist $H$ beschränkt, also gilt $\abs{H}\le b$ und folglich ist
\begin{align*}
\norm{[H\bullet \bar{N},H\bullet \bar{N}]_\infty^{1/2}}_{L^2}^2
= \E (H^2\bullet [\bar{N},\bar{N}])_\infty \le
b^2 \norm{[\bar{N},\bar{N}]_\infty^{1/2}}_{L^2}^2,\tag{1}
\end{align*}
sowie
\begin{align*}
\norm*{\int_0^\infty \abs{\ddd (H\bullet \bar{A})_s}}_{L^2}
\le b
\norm*{\int_0^\infty \abs{\ddd \bar{A}_s}}_{L^2}.\tag{2}
\end{align*}
Somit ist $H\bullet X\in\Hs^2$. Darüber hinaus stellen wir fest, dass sowohl
$H\bullet \bar{A}$ als auch $[N,N]$ von beschränkter Variation sind. Die
Integrale in (1) und (2) sind folglich als Lebesgue-Stieltjes-Integrale
erklärt, auch wenn $H$ nicht linksstetig sondern lediglich $\Pc$-messbar ist.
Wir versuchen daher, das Integral $H\bullet X$ auch für allgemeine Integranden
aus $\Pc$ zu erklären.

\subsection{Die Topologie der Integranden}

Wir definieren nun eine Topologie auf $\bPc$, welche an den Integrator $X$
angepasst ist, und zeigen, dass $\bL$ bezüglich dieser Topologie dicht in $\bPc$
liegt. Weiterhin können wir zeigen, dass bei einer Approximation von $H\in\bPc$
durch $H^n\in\bL$, die Integrale $H^n\bullet X$ eine Cauchyfolge in $\Hs^2$
bilden, welche aufgrund der Vollständigkeit konvergiert. Schließlich ist der
Grenzwert von $H^n\bullet X$ unabhängig von der gewählten Approximation, und das
Integral von $H\bullet X$ als dieser Grenzwert wohldefiniert.

\begin{definition}
\label{defn:4.2}
\nomenclature{$d_X$}{siehe}
Der Prozess $X\in\Hs^2$ besitze die kanonischen Zerlegung $X=\bar{N} +
\bar{A}$. Für $H,J \in \bPc$ wird durch 
\begin{align*}
d_X(H,J) \defl
\norm*{\left( \int_0^\infty (H_s-J_s)^2\,\ddd[\bar{N},\bar{N}]_s
\right)^{1/2}}_{L^2} + \norm*{\int_0^\infty |H_s - J_s| \,
|\ddd\bar{A}_s|}_{L^2}
\end{align*}
eine Metrik auf dem Raum $\bPc$ definiert.\fish
\end{definition}

\begin{rem*}
Die Metrik $d_X$ wird durch eine Norm induziert, welche man durch
\begin{align*}
\norm{H}_{d_X}\defl d_X(H,0).
\end{align*}
aus der Metrik zurück erhält.\map
\end{rem*}
 
\begin{theorem}
\label{prop:4.2}
Für $X \in \Hs^2$ ist $\bL$ dicht in $\bPc$ unter der Metrik $d_X$.\fish
\end{theorem}

Zum Beweis von Satz \ref{prop:4.2} verwendet folgendes Theorem über
monotone Klassen beschränkter Funktionen -- siehe
\cite{Protter:2004wfa,Dellacherie:1988tr}.

\begin{prop*}[Monotone-Klassen-Theorem]
\index{Monotone-Klassen-Theorem}
Sei $\Tc$ eine Menge, und $\Hc$ eine \emph{monotone Klasse}
beschränkter reellwertiger Funktionen auf $\Tc$, d.h.
\begin{defnenum}
\item $\Hc$ ist ein reeller Vektorraum,
\item $\Id_\Tc\in\Hc$, und
\item falls $0\le f_1\le f_2\le \ldots$ mit $f_n\in\Hc$, und
$f_\infty=\lim_n f_n$ punktweise existiert und beschränkt ist, so gilt
$f_\infty\in\Hc$.
\end{defnenum}
Ist nun $\Ms$ eine multiplikativ abgeschlossene Klasse reellwertiger
Funktionen auf $\Tc$ mit $\Ms \subset \Hc$, so enthält $\Hc$
alle beschränkten $\sigma(\Ms)$-messbaren Funktionen, wobei
\begin{align*}
\sigma(\Ms) \defl \sigma\left(\setdef{f^{-1}(B)}{f\in\Ms,\;
B\in\Bc}\right).\fish
\end{align*}
\end{prop*}

\begin{proof}[Beweis von Satz \ref{prop:4.2}]
Zu zeigen ist, dass für jedes $H\in\bPc$ und jedes $\ep > 0$ ein Prozess
$Y\in\bL$ existiert mit $d_X(H,Y) < \ep$. Wir definieren dazu
\begin{align*}
\Hc \defl \setdef{H\in\bPc}{\text{zu jedem }\ep > 0\text{ existiert ein
}Y\in\bL\text{ mit }d_X(H,Y) < \ep},
\end{align*}
und unser Ziel ist es zu zeigen, dass $\Hc=\bPc$ oder äquivalent,
dass $\Hc$ alle Prozesse $Y: \R_+\times \Omega\to\R$ enthält,
die messbar bezüglich der durch $\Ms\defl\bL$ auf $\Tc\defl\R_+\times \Omega$
erzeugten $\sigma$-Algebra sind. Sofern wir alle Voraussetzungen des monotone Klassen
Theorems verifizieren können, folgt die Behauptung.

Zunächst ist $\Ms=\bL$ eine multiplikative Klasse beschränkter Funktionen.
Weiterhin ist $\Hc$ ein reeller Vektorraum beschränkter Funktionen auf
$\Tc=\R_+\times \Omega$, mit $\Id_\Tc\in\Hc$, denn $\Id_\Tc\in\bL$. Sei
nun $H^1\le H^2 \le \ldots$ eine Folge von Prozessen in $\Hc$, welche
punktweise gegen einen beschränkten Prozess $H=\lim_{n\to \infty} H^n$
konvergiert. So ist $H$ ebenfalls $\bPc$-messbar, und da $\abs{H-H^n} \le
\abs{H}\le b$ eine integrierbare Majorante ist, folgt mit dem Satz von Lebesgue,
dass $d_X(H^n,H) \to 0$. 
Sei nun $\ep > 0$ beliebig, so existiert ein $n_\ep \ge 1$ mit
\begin{align*}
d_X(H,H^n) < \ep/2,\qquad n\ge n_\ep.
\end{align*}
Da jedes $H^n\in\Hc$ ist, können wir ein $n_0 \ge
n_\delta$ und ein $Y\in\bL$ wählen, so dass $d_X(H^n,J) < \ep/2$. Somit gilt
\begin{align*}
d_X(H,J) \le d_X(H,H^n) + d_X(H^n,J) < \ep,
\end{align*}
also ist $X\in\Hc$. Somit ist $\Hc$ monoton, und das
Monotone-Klassen-Theorem ist anwendbar. Folglich liegt $\bL$ dicht in
$\bPc$.\qed
\end{proof}

Die an den Integrator angepasste Topologie auf $\bPc$ haben wir gerade so
definiert, dass für eine Cauchyfolge $(H^n)$ unter $d_X$ die Integrale
$(H^n\bullet X)$ eine Cauchyfolge in $\Hs^2$ bilden. Hätten wir für alle
Integratoren eine einzige Topologie gewählt, wäre dies so nicht möglich
gewesen.

\begin{theorem}
\label{prop:4.3}
Sind $X \in \Hs^2$ und $(H^n)$ eine Cauchy-Folge unter $d_X$,
dann ist $(H^n \bullet X)$ eine Cauchy-Folge in $\Hs^2$.\fish
\end{theorem}
\begin{proof}
Offenbar gilt $\norm{H^n\bullet X-H^m\bullet X}_{\Hs^2} = d_X(H^n,H^m)$.\qed
\end{proof}

Approximieren wir nun einen Prozess $H\in\bPc$ durch eine Folge von Prozessen
$H\in\bL$, so bildet $H^n$ eine Cauchyfolge bezüglich $d_X$, und
$H^n\bullet X$ konvergiert in $\Hs^2$ aufgrund der Vollständigkeit. Um
das Integral von $H$ gegen $X$ als diesen Grenzwert definieren zu können, müssen
wir noch sicherstellen, dass der Grenzwert unabhängig von der gewählten
Approximation ist. 

\begin{theorem}
\label{prop:4.4}
Seien $X \in \Hs^2$ und $H\in \bPc$, ferner seien $(H^n)$ und
  $(J^m)$ zwei Folgen in $\bL$ mit $H^n \overset{d_X}{\longto} H$ und  $H^m
  \overset{d_X}{\longto} H$. Dann konvergieren $H^n\bullet X$ und $J^m\bullet X$
  gegen den selben Grenzwert in $\Hs^2$.\fish
\end{theorem}
\begin{proof}
Seien also $H^n$ und $J^m$ zwei Approximationen von $H$ in $\bL$, dann gilt nach
Satz \ref{prop:4.3}, dass
\begin{align*}
H^n\bullet X \Hto Y,\qquad J^m\bullet X \Hto Z.
\end{align*}
Zu jedem $\ep > 0$ existieren daher $n,m\ge 1$, so dass
\begin{align*}
\norm{Y-Z}_{\Hs^2} &\le \norm{Y-H^n\bullet X}_{\Hs^2} + \norm{H^n\bullet
X-J^m\bullet X}_{\Hs^2} + \norm{J^m\bullet X-Z}_{\Hs^2}\\
&\le 2 \ep + d_X(H^n,J^m)\\
&\le 2\ep + d_X(H^n,H) + d_X(J^m,H)
\le 4\ep.
\end{align*}
Also gilt $Y=Z$ und dies war zu zeigen.\qed
\end{proof}

\subsection{Erweiterung des Integrals}

Mit dieser Vorbereitung können wir nun das stochastische Integral auf
Integranden aus $\bPc$ ausdehnen.

\begin{definition}
\label{defn:4.3}
Sei $X$ ein Semimartingal in $\Hs^2$ und $H\in\bPc$. Ferner sei $(H^n)$ eine
Folge in $\bL$ mit $H^n \dto{X} H$. Dann ist das \emph{stochastische Integral $H
\bullet X$} definiert als das (eindeutige) Semimartingal
\begin{align*}
H\bullet X \defl \Hlim\limits_{n\to \infty} H^n\bullet X.
\end{align*}
Wir schreiben auch $H\bullet X = \Bigl(\int_0^t H_s\, \dX_s \Bigr)_{t \ge 0}
$.\fish
\end{definition}

Im Folgenden wollen wir möglichst viele Eigenschaften des
stochastischen Integrals aus Kapitel \ref{c:stoch-int} auch für dieses Integral
nachweisen. Folgender Satz wird uns dies an vielen Stellen erleichtern.

\begin{theorem}
\label{prop:4.5}
Für jedes Semimartingal $X\in\Hs^2$ gilt
\begin{align*}
\E (X^*)^2 \le 8 \|X\|^2_{\Hs^2},\qquad X^* = \sup_{t\ge 0} |X_t|.\fish 
\end{align*}
\end{theorem}
\begin{proof}
Sei $X = \bar{N}+\bar{A}$ die kanonische Zerlegung, so gilt $X^* \le \bar{N}^*
+ \bar{A}^*$. Offenbar gilt $\abs{A_t}\le \int_0^t \abs{\dA_s}$ für alle $t\ge
0$, also gilt dies auch für das Supremum. Weiterhin ist $\bar{N}$ ein
$L^2$-Martingal, daher können wir die $L^2$-Ungleichung \ref{prop:1.19} von Doob
anwenden und erhalten die Abschätzung
\begin{align*}
\E (\bar{N}^*)^2 \le 4 \E \bar{N}_\infty^2 =
4\norm{[\bar{N},\bar{N}]_\infty^{1/2}}_{L^2}
\end{align*}
Zusammengenommen ergibt sich
\begin{align*}
\E (X^*)^2 &\le 2 \left(\E (\bar{N}^*)^2 + \E (\bar{A}^*)^2\right)\\
&\le 8 \left( \norm*{[\bar{N},\bar{N}]_\infty^{1/2}}_{L^2}^2 +
\norm*{\int_0^\infty \abs{\dA_s}}_{L^2}^2\right)\le 8\norm{X}_{\Hs^2}^2.\qed
\end{align*}
\end{proof}

Als unmittelbare Folgerung erhalten wir, dass Konvergenz in $\Hs^2$ f.s.
gleichmäßige Konvergenz in $t$ einer Teilfolge ergibt.

\begin{korollar}
\label{cor:4.1}
Seien $X^n,X\in\Hs^2$ mit $X^n \Hto X$. Dann existiert eine Teilfolge
$(X^{n_k})$ mit 
\begin{align*}
(X^{n_k} - X)^* \fsto 0.\fish
\end{align*}
\end{korollar}
\begin{proof}
Nach Satz \ref{prop:4.5} gilt $\E (X^n-X)^* \to 0$, also existiert eine
Teilfolge $(n_k)$, so dass $(X^{n_k}-X)\fsto 0$.\qed
\end{proof}


Viele der in Kapitel \ref{c:stoch-int} für das Integral mit Integranden
$H\in\bL$ geltenden Eigenschaften gelten auch für Integrale mit  Integranden
$H\in \bPc$.

\begin{theorem}[Linearität des Integrals]
\label{asso}
\label{prop:4.6}
Seien $X,Y \in \Hs^2$ und $H,K \in \bPc$. Dann gelten
\begin{align*}
(H+K) \bullet X &= H \bullet X + K \bullet X,\\
H \bullet (X+Y) &= H \bullet X + H \bullet Y.\fish
\end{align*}
\end{theorem}
\begin{proof}
Sowohl $\Hs^2$ als auch $\bPc$ sind lineare Räume. Die Linearität des
Integrals im Integranden ist offensichtlich. Sei nun $H^n$ eine Folge in $\bL$
mit $H^n\dto{X+Y} H$. Dann gilt auch $H^n\dto{X}H$ und $H^n\dto{Y} H$, so dass
die Linearität im Integrator folgt.\qed
\end{proof}

\begin{theorem}
\label{prop:4.7}
Sei $T$ eine Stoppzeit. Dann gilt
\begin{align*}
(H \bullet X)^T = H\Id_{(0,T]} \bullet X = H \bullet (X^T).\fish
\end{align*}
\end{theorem}
\begin{proof}
Sei $T$ eine Stoppzeit, dann ist $\Id_{(0,T]}\in \bL\subset \bPc$, und
$X^T$ ist ein spezielles Semimartingal mit $X^T\in\Hs^2$. Sei nun $(H^n)$
eine Folge in $\bL$ mit $H^n\dto{X} H$, dann gilt
\begin{align*}
(H^n\bullet X)^T = (H^n\Id_{(0,T]})\bullet X = H^n\bullet X^T.
\end{align*}
Weiterhin gilt $H^n\bullet X^T \Hto H\bullet X$, und da
$d_X(H^n\Id_{(0,T]},H\Id_{(0,T]}) \le d_X(H^n,H)$, gilt auch
$(H^n\Id_{(0,T]})\bullet X\Hto (H\Id_{(0,T]})\bullet X$. Letztlich gilt aufgrund
der Linearität des Integrals
\begin{align*}
(H^n\bullet X)^T - (H\bullet X)^T &= ((H^n-H)\bullet X)^T \\ &= ((H^n-H)\bullet
\bar{N})^T + ((H^n-H)\bullet \bar{A})^T.
\end{align*}
Also folgt für die Norm
\begin{align*}
\norm{((H^n-H)\bullet X)^T}_{\Hs^2} &=
\norm{[(H^n-H)\bullet X,(H^n-H)\bullet X]_T^{1/2}}_{L^2}\\
&\quad\; +
\norm{\int_0^T \abs{H^n-H}_s \abs{\dA_s}}_{L^2}\\
&\le \norm{(H^n-H)\bullet X}_{\Hs^2} \to 0.\qed 
\end{align*}
\end{proof}

Auch bezüglich Regularität verhält sich das erweiterte stochastische Integral
>>wie üblich<< -- allein die Unstetigkeitsstellen des Integrators bestimmen die
Unstetigkeitsstellen des Integrals.

\begin{theorem}
\label{prop:4.8}
Seien $H \in \bPc$ und $X \in \Hs^2$. Der Sprungprozess $\left( \Delta
    (H\bullet X)_s \right)_{s\ge 0} $ ist ununterscheidbar von $\left(H\bullet
    (\Delta X_s) \right)_{s\ge 0}$.\fish
\end{theorem}
\begin{proof}
Folgt unter Verwendung von Satz \ref{prop:2.11} und den Konvergenzeigenschaften
des Integrals.\fish
\end{proof}

Wir benötigen insbesondere folgende Version von Satz \ref{prop:4.7} für die
linksstetige Version des gestoppten Integrators.

\begin{korollar}
\label{cor:4.2}
Seien $H \in \bPc$, $X \in \Hs^2$ und $T$ eine endliche Stoppzeit. Dann gilt
\begin{align*}
H \bullet (X^{T-}) = (H \bullet X)^{T-}.\fish
\end{align*}
\end{korollar}

Für Integratoren $X$ von beschränkter Variation ist das Integral $H\bullet X$
auch ohne den in diesem Kapitel entwickelten Unterbau bereits als
Lebesgue-Stieltjes-Integral erklärt. Darüber hinaus stimmt es mit dem in diesem
Abschnitt konstruierten Integral überein.

\begin{theorem}
\label{prop:4.9}
Seien $X \in \Hs^2$ mit Pfaden von beschränkter Variation auf kompakten
 Mengen und $H \in \bPc$. Dann stimmt $H \bullet X$ mit dem pfadweise
  definierten Lebesgue-Stieltjes-Integral überein.\fish
\end{theorem}

Auch für Satz \ref{prop:2.13} existiert eine direkte Verallgemeinerung.

\begin{theorem}[Assoziativität]
\label{prop:4.10}
Seien $X \in \Hs^2$ und $H,K \in \bPc$. Dann gilt $ K \bullet X \in \Hs^2$ und
\begin{align*}
  H \bullet (K \bullet X) = (HK) \bullet X.\fish
\end{align*}  
\end{theorem}

Während das stochastische Integral bezüglich eines Martingals im allgemeinen
lediglich ein lokales Martingal darstellt, gilt für den Fall von Integratoren
aus $\Hs^2$ deutlich mehr.

\begin{theorem}
\label{prop:4.11}
  Seien $X\in \Hs^2$ ein Martingal (und damit ein quadratintegrierbares
  Martingal) und $H \in  \bPc$. Dann ist auch $H \bullet X$ ein
  quadratintegrierbares Martingal.\fish
\end{theorem}

Mit dem folgenden Satz -- ein Analogon zu Satz \ref{prop:2.19} -- lässt
sich die Berechnung des Kovariationsprozesses zweier stochastischer Integrale
auf eine Integration bezüglich des Kovariationsprozesses der Integratoren
zurückführen. Diese Identität ermöglicht es in vielen Anwendungen, stochastische
Integrale tatsächlich zu berechnen.

\begin{theorem}
\label{itoident}
Seien $X,Y \in \Hs^2$ und $H,K \in \bPc$. Dann gilt
\begin{align*}
[H \bullet X, K \bullet Y]_t  = \int_0^t H_sK_s\,\ddd[X,Y]_s,\qquad t\ge
0.\fish
\end{align*}
\end{theorem}


\section{Integration vorhersagbarer Prozesse}

Ziel dieses Abschnittes ist es, die Klasse der Integranden von $\bPc$
ausgehend nochmals zu erweitern, indem wir auch unbeschränkte vorhersagbare
Prozesse zulassen, welche einer gewissen Integrierbarkeitsbedingung genügen --
ähnlich zu den $L^1(\mu)$-Funktionen im Fall des Lebesgue-Integrals. Darüber
hinaus wollen wir die Einschränkung bezüglich der Integratoren auf
$\Hs^2$-Semimartingale eliminieren, die wir am Anfang des Abschnittes gemacht
haben, um das Integral zu verallgemeinern.

Zunächst zitieren wir ein technisches Lemma. 

\begin{lemma}
\label{lem:4.2}
Sei $A$ ein FV Prozess mit $A_0=0$ und $\int_0^\infty |\dA_s| \in L^2$.
Dann gilt $A \in \Hs^2$ und 
\begin{align*}
\norm{A}_{\Hs^2} \le 6 \norm*{\int_0^\infty |\dA_s|}_{L^2}.\fish
\end{align*}
\end{lemma}

Man beachte, dass im Allgemeinen unter obigen Voraussetzungen $A = \bar{A}$
\textit{nicht} gilt, d.h. $A = \bar{N} + \bar{A}$ wobei der lokale Martingalteil
$\bar{N}$ ein von Null verschiedener reiner Sprungprozess ist. Im Besonderen
ist $A$ im Allgemeinen nicht vorhersagbar, andernfalls wäre ohnehin $\bar{A} = A$.

Ein Semimartingal ist ohne Zusatzannahmen kein spezielles Semimartingal, nicht
einmal lokal. Insbesondere ist daher ein typisches Semimartingal kein lokales
$\Hs^2$-Semimartingal. Um letzteres zu erreichen, müssen wir die Definition der
Lokalität weiter abschwächen.
% 
% Ist $X$ ein Prozess mit $X_0=0$, so gilt eine Eigenschaft $\pi$ lokal für $X$,
% falls es eine Folge von Stoppzeiten $(T_n)$ gibt mit $0=T^0 \le T^1 \le T^2
% \ldots $ und $ \lim_n T^n=\infty$ f.s., so dass $X^{T^n}$ für jedes $n$ die
% Eigenschaft $\pi$ besitzt.

\begin{definition}
\label{defn:4.4}
\index{prälokal}
Sei $X$ ein Prozess mit $X_0=0$. Eine Eigenschaft $\pi$ gilt \emph{prälokal}
für $X$, falls eine Fundamentalfolge von Stoppzeiten $(T_n)$ existiert, so dass
$X^{T^n-}$ für jedes $n$ die Eigenschaft $\pi$ besitzt, wobei
\begin{align*}
X_t^{T-}\defl X_t\Id_{[0,T)}(t) + X_{T-}\Id_{[T,\infty)}(t),\qquad t\ge 0.\fish
\end{align*}
\end{definition}

Grob gesagt, wird der Prozess $X$ nicht zum zufälligen Zeitpunkt $T$, sondern zu
einem infinitesimal früheren Zeitpunkt $T-$ abgestoppt. Dadurch gewinnen wir
zusätzliche Regularitätseigenschaften.

\begin{theorem}
\label{prop:4.13}
Jedes Semimartingal ist ein prälokales $\Hs^2$-Semimartingal.\fish
\end{theorem}
\begin{proof}
Wir müssen zeigen, dass eine Fundamentalfolge von Stoppzeiten $(T^n)$
existiert, so dass $X^{T^n-}\in \Hs^2$ für jedes $n$. Aus der
Semimartingaleigenschaft von $X$ folgt, dass $X = M + A$ mit einem lokalen
Martingal $M$ und einem FV-Prozess $A$. Nach dem Fundamentalsatz für lokale
Martingale \ref{prop:3.1} können wir dabei für $\beta > 0$ beliebig die Zerlegung so
wählen, dass $\abs{\Delta M} \le \beta$. Setzen wir nun
\begin{align*}
T^n \defl \inf\setdef{t > 0}{[M,M]_t > 0 \text{ oder } \int_0^t \abs{\dA_s} >
n},
\end{align*}
so ist $(T^n)$ eine Fundamentalfolge. Fixieren wir ein $n\ge 1$ und betrachten
$Y \defl X^{T_n-}$, so gilt $\abs{\Delta Y} \le \beta + 2n$. Also ist $Y$
ein Semimartingal mit beschränkten Sprüngen und folglich nach Satz
\ref{prop:3.6} ein spezielles Semimartingal, d.h. $Y$ besitzt eine eindeutige
Zerlegung in ein lokales Martingal und einen vorhersagbaren FV Prozess.
Andererseits ist
\begin{align*}
Y &= M^{T_n-} + A^{T_n-} = M^{T_n} + A^{T_n-} - (M^{T_n}-M^{T_n-}) \\
&= M^{T_n}
+ C,\qquad C \defl A^{T_n-} - \Delta M_{T_n}\Id_{[T_n,\infty)}.
\end{align*}
Unabhängig von $t\ge 0$ gilt nun
\begin{align*}
[M^{T_n},M^{T_n}]_t = 
[M^{T_n-},M^{T_n-}] + (\Delta M_{T_n})^2
\le n + \beta^2, 
\end{align*}
also ist $M^{T_n}$ nach Satz \ref{prop:2.18} sogar ein $L^2$-Martingal und
folglich in $\Hs^2$. Weiterhin ist die Totalvariation von $C$ beschränkt durch
\begin{align*}
\int_0^\infty \abs{\ddd C_s} \le n + \abs{\Delta M_{T_n}} \le \beta + n.
\end{align*}
Somit ist $C$ von beschränkter Variation, und da $A^{T_n-}\in\L$ und
$\Id_{[T_n,\infty)}$ adaptiert ist, ist $C$ sogar vorhersagbar und damit in
$\Hs^2$. Deshalb ist $Y = X^{T_n-}$ in $\Hs^2$ und $X$ ist ein prälokales
$\Hs^2$-Semimartingal.\qed
\end{proof}

Im vergangenen Abschnitt haben wir die Beschränktheit der Integranden nur
dazu benötigt, dass die Integrale (1) und (2) endlich sind. Wenn wir dies
einfach fordern, so sind auch unbeschränkte Integranden zulässig.

\begin{definition}
\label{defn:4.5}
\index{$(\Hs^2,X)$-integrierbar}
Sei $X \in \Hs^2$ mit kanonischer Zerlegung
  $X=\overline{N}+\overline{A}$. Ein Prozess $H \in \Pc$
  heißt \emph{$(\Hs^2,X)$-integrierbar}, falls
\begin{align*}
\E\int_0^\infty H^2_s\,\ddd[\overline{N},\overline{N}]_s +
\E\left( \int_0^\infty |H_s| \,|\ddd\overline{A}_s| \right)^2 < \infty.\fish
\end{align*}
\end{definition}

Man macht sich sofort klar, dass die $(\Hs^2,X)$-integrierbaren Prozesse einen
metrischen Raum bilden bezüglich der Metrik $d_X$ aus Definition \ref{defn:4.2},
und dass diese Metrik durch eine Norm induziert wird. Es verbleibt nur noch zu
zeigen, dass man jeden $(\Hs^2,X)$-integrierbaren Prozess in der $d_X$-Metrik
durch beschränkte vorhersagbare Prozesse approximieren kann. 

\begin{theorem}
\label{prop:4.14}
Sei $X\in\Hs^2$ ein Semimartingal und $H\in\Pc$ ein $(\Hs^2,X)$-integrierbarer
Prozess. Ferner sei $H^n$ eine Folge in $\bPc$ mit $H^n\fsto H$ und
$\abs{H^n}\le \abs{H}$ für $n\ge 1$, dann ist $H^n \bullet X$ eine Cauchy-Folge
in $\Hs^2$, und ihr Grenzwert ist unabhängig von der gewählten Folge $H^n$.\fish
\end{theorem}
\begin{proof}
Nach Definition der $\Hs^2$-Norm \ref{defn:4.1} ist
\begin{align*}
d_X(H^n,H^m) &= \norm{H^n\bullet X - H^m\bullet X}_{\Hs^2}\\
&= 
\norm*{\left(\int_0^\infty
(H^n_s-H^m_s)\,\ddd[\bar{N},\bar{N}]_s\right)^{1/2}}_{L^2}
+
\norm*{\int_0^\infty
\abs{H^n_s-H^m_s}\,\abs{\ddd\bar{A}_s}}_{L^2}.
\end{align*}
Ferner gilt $H^n-H^m \fsto 0$ für $n,m\to \infty$ sowie $\abs{H^n-H^m}\le
2\abs{H}$, und die Integratoren $[\bar{N},\bar{N}]_s$ und $\abs{\bar{A}}_s$ sind
von beschränkter Variation. Also sind die Integrale als
Lebesgue-Stieltjes-Integrale erklärt, und wir können den Satz von der
dominierten Konvergenz anwenden, um die Cauchy-Eigenschaft zu erhalten.
Die Unabhängigkeit des Grenzwertes von der gewählten Folge $H^n$ folgt dann
genau wie in Satz \ref{prop:4.4}.\qed
\end{proof}

Nun können wir durch geeignete Approximation das Integral auch für
unbeschränkte Integranden definieren.

\begin{definition}
\label{defn:4.6}
Sei $X$ ein Semimartingal in $\Hs^2$ und $H\in\Pc$ $(\Hs^2,X)$-integrierbar.
Dann ist das stochastische Integral $H \bullet X$ definiert durch
\begin{align*}
H\bullet X \defl \Hlim\limits_{n\to \infty}\; H^n\bullet X,
\end{align*}
mit $H^n\fsto H$ und $H^n\in\bPc$.\fish
\end{definition}

Somit haben wir die Klasse der Integranden von den linksstetigen Prozessen $\L$
auf die wesentlich größere Klasse der vorhersagbaren Prozesse $\Pc$ vergrößert,
allerdings mussten wir uns dazu auf $\Hs^2$-Semimartingale als Integratoren
zurückziehen. Im letzten Schritt wollen wir uns dieser Einschränkung wieder
entledigen. Dazu verwenden wir Satz \ref{prop:4.13} nach dem jedes Semimartingal
ein prälokales $\Hs^2$-Semimartingal ist.

\begin{definition}
\label{defn:4.7}
Sei $X$ ein Semimartingal und $H\in\Pc$. Falls es eine
Fundamentalfolge von Stoppzeiten $(T_n)$ gibt, so dass $X^{T^n-}\in \Hs^2$
für jedes $n \ge 1$ und $H$ $(\Hs^2,X^{T^n-})$-integrierbar ist, so ist das
stochastische Integral $H \bullet X$ definiert durch
\begin{align*}
H \bullet X \defl H \bullet (X^{T^n-}) \text{ auf $[0,T^n)$}.
\end{align*}
In diesem Fall heißt $H$ $X$-integrierbar, kurz $H \in L(X)$.\fish
\end{definition}

Wir müssen noch zeigen, dass das so definierte stochastische Integral $H \bullet
X$ nicht von der Wahl der Folge von Stoppzeiten abhängt.

\begin{proof}[Nachweis der Wohldefiniertheit]
Sei $T$ eine Stoppzeit, so dass $X^{T_-}\in\Hs^2$ und $H$
$(X^{T_-},\Hs^2)$-integrierbar ist. Weiter gelte $H^k\fsto H$ mit $H^k\in
\bPc$. Sei $S\le T$ eine weitere Stoppzeit, dann gilt nach Korollar
\ref{cor:4.2}, dass
\begin{align*}
(H^k\bullet X^{T-})^{S-} = 
H^k\bullet (X^{(T\wedge S)-})
= H^k \bullet X^{S-},\qquad k\ge 1.
\end{align*}
Linke und rechte Seite bilden Cauchyfolgen in $\Hs^2$ und der Grenzwert hängt
nicht von $H^k$ ab, also gilt auch $(H\bullet X^{T-})^{S-} = H\bullet X^{S-}$.

Seien nun $T^n$ und $S^n$ zwei Fundamentalfolgen, so dass $X^{T_n-}$, 
$X^{S_n-}\in\Hs^2$ und $H$ sowohl $(X^{T_n-},\Hs^2)$ als auch
$(X^{S_n-},\Hs^2)$-integrierbar ist. Nach obiger Rechnung gilt dann
\begin{align*}
(H\bullet X^{T_n-})^{S_n-} = 
H\bullet (X^{(T_n\wedge S_n)-}) =
(H\bullet X^{S_n-})^{T_n-}, 
\end{align*}
folglich stimmen $H\bullet X^{T_n-}$ und $H\bullet X^{S_n-}$ auf $[0,T_n\wedge
S_n)$ überein, und das Integral ist wohldefiniert.\qed
\end{proof}

Offenbar gilt $\bPc \subset L(X)$ für jedes Semimartingal $X$. Die Klasse $L(X)$
ist aber wesentlich größer, denn auch lediglich lokal beschränkte Integranden
sind zulässig.
% Gemäß des Konzepts der Lokalisierung heißt ein Prozess $H$ lokal beschränkt,
% falls es eine Folge von Stoppzeiten $(T_n)$ mit $0=T^0 \le T^1 \le T^2 \le
% \ldots $ und $\lim_n T^n=\infty$ f.s.\ gibt, so dass $H^{T^n}1_{[T^n>0]}$ für
% jedes $n$ ein beschränkter Prozess ist. Jeder Prozess $H\in\mbbL$ ist lokal
% beschränkt.

\begin{theorem}
\label{prop:4.15}
Ist $X$ ein Semimartingal und $H \in \Pc$ lokal beschränkt, dann existiert
das stochastische Integral $H \bullet X$, kurz $H\in L(X)$.\fish
\end{theorem}
\begin{proof}
Seien $(S^n)$ und $(T^n)$ zwei Fundamentalfolgen von Stoppzeiten, so dass
$(H^{S^n}\Id_{[T^n > 0]})$ für alle $n\ge 1$ beschränkt ist, und
$X^{T_n-}$ in $\Hs^2$ liegt. Definieren wir nun $R^n \defl S^n\wedge
T^n$, so ist $(R^n)$ eine Fundamentalfolge für die $Y \defl H^{R^n}\Id_{[R^n >
0]}$ auf $(0,R^n)$ beschränkt ist, und ferner $\dX^{R^n-} = 0$ außerhalb von
$(0,R^n)$ gilt. Also ist $Y$ $(\Hs^2,X^{R^n-})$-integrierbar, und folglich ist
das Integral $H\bullet X$ gemäß Definition \ref{defn:4.7} erklärt.\qed
\end{proof}

Man kann zeigen, dass die in den Sätzen \ref{asso}--\ref{itoident} des letzten
Abschnitts genannten Eigenschaften in sinngemäßer Weise auch für das eben
definierte stochastische Integral gelten.

\begin{theorem}[Linearität im Integranden]
\label{prop:4.16}
Seien $X$ ein Semimartingal und
$H,J\in L(X)$.
Dann gelten
\begin{align*}
\alpha H + \beta J &\in L(X),\qquad\text{und}\qquad
(\alpha H + \beta J) \bullet X = \alpha H\bullet X + \beta J \bullet X.
\end{align*}
Also ist $L(X)$ ein linearer Raum.\fish
\end{theorem}

\begin{theorem}[Linearität im Integrator]
\label{prop:4.17}
Seien $X,Y$ Semimartingale und $H \in
L(X)$ und $H \in L(Y)$. Dann gelten
\begin{align*} H &\in L(X+Y),\qquad\text{und}\qquad
H \bullet (X+Y) = H \bullet X + H \bullet Y.\fish
\end{align*}
\end{theorem}

\begin{theorem}
\label{prop:4.18}
Sei $X$ ein Semimartingal und $H \in L(X)$. So ist der
  Sprungprozess $\left( \Delta (H\bullet X)_s \right)_{s\ge 0}$
  ununterscheidbar von $\left(H\bullet (\Delta X_s) \right)_{s\ge 0} $.
\end{theorem}

\begin{theorem}
\label{prop:4.19}
Seien $T$ eine Stoppzeit, $X$ ein Semimartingal und $H \in
  L(X)$. Dann gilt
\begin{align*}
(H \bullet X)^T \;= H1_{[0,T]} \bullet X = H \bullet \Bigl(X^T\Bigr).
\end{align*}
Ferner gilt, mit der Konvention $\infty-$ gleich $\infty$,
\begin{align*}
(H \bullet X)^{T-} = H \bullet \left(X^{T-}\right).\fish
\end{align*}
\end{theorem}

\begin{theorem}
\label{prop:4.20}
  Seien $X$ ein Semimartingal mit Pfaden von beschränkter Variation auf
  kompakten Mengen und $H \in L(X)$, so dass das Lebesgue-Stieltjes-Integral
  $\int_0^t |H_s| |\dX_s| $ f.s.\ existiert. Dann stimmt das stochastische
  Integral  $H \bullet X$ mit dem
  pfadweise definierten Lebesgue-Stieltjes-Integral überein.\fish
\end{theorem}

\begin{theorem}[Assoziativität]
\label{prop:4.21}
  Sei $X$ ein Semimartingal und $K \in L(X)$. Dann gilt $H\in L(K\bullet X)$
  genau dann, wenn $HK \in L(X)$. In diesem Fall gilt auch
\begin{align*}
H \bullet (K \bullet X) = (HK) \bullet X.\fish
\end{align*}
\end{theorem}

\begin{theorem}
\label{prop:4.22}
Seien $X,Y$ zwei Semimartingale, sowie $H \in L(X)$ und $K \in L(Y)$. Dann gilt
\begin{align*}
[H \bullet X, K \bullet Y]_t  = \int_0^t H_sK_s\,\ddd[X,Y]_s,\qquad t\ge 0.
\end{align*}
\end{theorem}

Auch für unbeschränkte vorhersagbare Integranden aus $L(X)$ ist das
stochastische Integral abgeschlossen unter quadratintegrierbaren Martingalen.

\begin{lemma}
\label{lem:4.3}
Ist $M$ ein quadratintegrierbares Martingal und $H \in \Pc$ mit
\begin{align*}
\E\int_0^\infty H^2_s \, \ddd[M,M]_s < \infty, 
\end{align*}
dann ist $H \bullet M$ ein quadratintegrierbares Martingal.\fish
\end{lemma}
\begin{proof}
Sei $H^k \defl H\cdot \Id_{[\abs{H}\le k]}$, dann ist $H^k\in\bPc$ mit $H^k\fsto
H$ und $H^k\bullet M$ ist nach Satz \ref{prop:4.11} ein quadratintegrierbares
Martingal. Außerdem ist $(H^k\bullet M)$ eine $\Hs^2$-Cauchyfolge mit
$H^k\bullet M\Hto H\bullet M$. Insbesondere gilt daher $H^k\bullet M\lto{1}
H\bullet M$, und folglich ist $H\bullet M$ ein Martingal.\qed
\end{proof}

Wir können damit zeigen, dass auch die Klasse der lokal quadratintegirerbaren
lokalen Martingale abgeschlossen unter Integration ist.

\begin{theorem}
\label{prop:4.23}
Sei $M$ ein lokal quadratintegrierbares lokales Martingal und $H \in \Pc$.
Falls eine Fundamentalfolge von Stoppzeiten $(T^n)$ existiert, so dass 
\begin{align*}
\E\int_0^{T^n} H^2_s \, \ddd[M,M]_s < \infty,\qquad n\ge 1,
\end{align*}
dann existiert das stochastische Integral $H \bullet X$, d.h.\ $H\in L(X)$, und
ist ein lokal quadratintegrierbares lokales Martingal.\fish
\end{theorem}
\begin{proof}
Aus den Voraussetzungen folgt, dass eine Fundamentalfolge $(\tilde T^n)$ von
Stoppzeiten existiert, so dass $M^{\tilde T^n}$ ein quadratintegrierbares
Martingal ist, und weiter gilt
\begin{align*}
\E\int_0^{T^n} H^2_s \, \ddd[M^{\tilde T^n},M^{\tilde T^n}]_s < \infty,\qquad
n\ge 1.
\end{align*}
Mit Lemma \ref{prop:4.3} folgt, dass $H\bullet M^{\tilde T^n}$ ein
quadratintegrierbares Martingal ist. Also ist $H\bullet M$ ein lokal
quadratintegrierbares lokales Martingal.\qed
\end{proof}

Als unmittelbares Korollar erhalten wir folgende Aussage.

\begin{theorem}
Ist $M$ ein lokales Martingal und $H \in \Pc$ lokal beschränkt, dann ist das
stochastische Integral $H \bullet M$ ein lokales Martingal.
\end{theorem}

Die lokale Quadratintegrierbarkeit bzw. lokale Beschränktheit als Voraussetzung
für die Abgeschlossenheit der Integration unter lokalen Martingalen lässt sich
jedoch nicht wesentlich weiter abschwächen. Für ein lokales Martingal $M$ und
einen Prozess $H\in L(X)$ ist das stochastische Integral $H \bullet M$ im
Allgemeinen nämlich kein lokales Martingal, wie das Gegenbeispiel von Emery
eindrücklich zeigt.

Die Klasse der lokal beschränkten vorhersagbaren Prozesse ist im Wesentlichen
die größte Klasse von Integranden auf die sich die stochastische Integration
ausdehnen lässt, ohne die zentrale Eigenschaft der Abgeschlossenheit unter
Integration bezüglich lokaler Martingale zu verlieren.