\chapter{Martingaldarstellungssätze}

Ein interessantes Problem in den Anwendungen ist die Fragestellung, wann man
einen gegeben Prozess als stochastisches Integral bezüglich einfacherer
Prozesse darstellen kann. Dieses Problem umfasst beispielsweise die
Frage nach der Hedgebarkeit des Werteprozesses eines dynamischen Portfolios.
Wenn sich dieser Prozess nicht in geeigneter Weise als stochastisches Integral
darstellen lässt, so riskiert die ausstellende Seite bei ungünstigen
Kursverläufen große Verluste in Kauf zu nehmen.

\section{Der Raum der $L^2$-beschränkten Martingale}

Der gefilterte Wahrscheinlichkeitsraum $(\Omega, \F, \Fc,P)$ erfülle wieder
die üblichen Bedingungen. Wir betrachten zwei Martingale $L$ und $M$ und
untersuchen die Fragestellung, unter welchen Voraussetzungen es einen
stochastischen Prozess $H$ gibt, so dass 
\begin{align*}
L = H\bullet M.
\end{align*}
Um technische Schwierigkeiten zu vermeiden, beschränken wir uns auf
$L^2$-Martingale, denn für deren Analyse stehen uns Hilbertraummethoden zur
Verfügung.

\begin{definition}
\label{defn:5.1}
\nomenclature[S]{$\Ms^2$}{siehe}
\nomenclature[C]{$X_n\Mto X$}{$X_n\to X$ in $\Ms^2$}
\index{$\Ms^2$-Norm}
Der \emph{Raum der $L^2$-beschränkten
Martingale} ist definiert durch
\begin{align*}
\Ms^2 \defl \setdef{M}{M\text{ ist ein $L^2$-Martingal mit }M_0 = 0\;\fs},
\end{align*}
und durch
\begin{align*}
\norm{M} \defl (\E M_\infty^2)^{1/2}= (\E[M,M]_\infty )^{1/2}
\end{align*}
ist eine Norm auf $\Ms^2$ gegeben, durch die $\Ms^2$ zu einem Hilbertraum
wird.\fish
\end{definition}

Der Raum $\Ms^2$ ist ein abgeschlossener Teilraum von $\Hs^2$, genauer gilt
\begin{align*}
\Hs^2 = \Ms^2 \oplus \Ac,\qquad \Ac\defl \setd*{A\text{ ist FV,
vorhersagbar und $\norm*{\int_0^\infty \abs{\dA_s}}_{L^2} < \infty$}},
\end{align*}
und die Norm auf $\Ms^2$ ergibt sich durch Einschränkung der $\Hs^2$-Norm.
Insbesondere ist $\Ms^2$ vollständig bezüglich dieser Norm -- siehe Satz
\ref{prop:4.1}.

Man verifiziert leicht, dass $\norm{\cdot}$ die Parallelogrammgleichung erfüllt
und folglich durch ein Skalarprodukt induziert wird. Letzteres ist gegeben durch
\begin{align*}
\lin{N,M}_{\Ms^2} \defl (\E N_\infty M_\infty)^{1/2}.
\end{align*}
Somit ist $\Ms^2$ ein vollständiger Innenproduktraum, also ein Hilbertraum.

\subsection{Stabilität}

\begin{definition}
\label{defn:5.2}
\index{Stabiler Unterraum}
Ein abgeschlossener Unterraum $F$ von $\Ms^2$ heißt \emph{stabiler Unterraum},
falls er unter Stoppen stabil ist, d.h.\ für alle $M\in F$ und jede Stoppzeit
$T$ gilt $M^T \in F$.\fish
\end{definition}

Der folgende Satz liefert uns eine nützliche Charakterisierung stabiler
Unterräume.

\begin{theorem}
\label{prop:5.1}
Sei $F$ ein abgeschlossener Unterraum von $\Ms^2$. Dann sind die folgenden
Behauptungen äquivalent:
\begin{enumerate}
  \item Für jedes $M\in F$ und für alle $t \ge 0$ und $\Lambda\in\Fc_t$ gilt
  \begin{align*}
  (M-M^t) 1_\Lambda \in F.
  \end{align*}
  \item $F$ ist ein stabiler Unterraum von $\Ms^2$.
  \item Für jedes $M\in F$ und jeden Prozess $H\in\bPc$ gilt $ H\bullet M \in
  F$.
  \item Für jedes $M\in F$ und jeden Prozess $H\in\Pc$ mit
    $\E\int_0^\infty H^2_s\, \ddd[M,M]_s < \infty$ gilt $ H\bullet M \in
    F$.\fish
\end{enumerate}
\end{theorem}
\begin{proof}
(c)$\Rightarrow$(b): Sei $M\in F$ und $T$ eine Stoppzeit. Setzen wir $H =
\Id_{[0,T]}$, so ist $H$ vorhersagbar und beschränkt, und
\begin{align*}
M^T - M_0 = \int_0^T \dM = H\bullet M \in F.
\end{align*}
Da $M_0 = 0$ folgt $M^T \in F$, also ist $F$ stabil.

(b)$\Rightarrow$(a): Sei $M\in F$, $t\ge 0$ und $\Lambda\in\Fc_t$. Setzen wir
nun
\begin{align*}
T \defl t_\Lambda \defl
\begin{cases}
t, & \omega\in\Lambda,\\
\infty, & \text{sonst},
\end{cases}
\end{align*}
so ist $T$ eine Stoppzeit, und es gilt
\begin{align*}
(M-M^t)\Id_{\Lambda} = M-M^T \in F,
\end{align*}
denn $M\in F$ und $M^T\in F$.

(a)$\Rightarrow$(d): Sei $H$ zunächst von folgender Gestalt
\begin{align*}
H = \Id_{\Lambda_0}\Id_{\setd{0}} + \sum_{1\le i\le n} \Id_{\Lambda_i}
\Id_{(t_i,t_{i+1}]},\qquad \Lambda_{i}\in\Fc_{t_i},
\end{align*}
wobei $0 = t_0 \le t_1 \le \cdots \le t_n < \infty$. Sei $M\in F$, dann ist
$H\bullet M\in F$, denn
\begin{align*}
H\bullet M = \sum_{1\le i\le n} \Id_{\Lambda_i}(M^{t_{i+1}}-M^{t_i}),
\end{align*}
und $\Id_{\Lambda_i}(M^{t_{i+1}}-M^{t_i}) \in F$ nach (a), also auch jede
Linearkombination. Die Prozesse $H$ von obiger Gestalt liegen dicht in $\bL$ und
$\bL$ liegt seinerseits dicht in $\bPc$ bezüglich $d_X$. Ferner liegt $\bPc$
dicht in der Menge
\begin{align*}
\setdef{H\in\Pc}{\E\int_0^\infty H_s^2 \,\ddd[M,M]_s < \infty}.
\end{align*}
Für jedes solche $H$ existiert daher eine Approximation $H^k$ mit $H^k\bullet
M\in F$ und $(H^k\bullet M)$ Cauchyfolge in $\Hs^2$, welche gegen $H\bullet M$
konvergiert. Nach Voraussetzung ist $F$ abgeschlossen, also liegt auch
$H\bullet M$ in $F$.\qed
\end{proof}

Da beliebige Durchschnitte abgeschlossener stabiler Unterräume wieder
stabil und abgeschlossen sind, ist die folgende Definition sinnvoll.

\begin{definition}
\label{defn:5.3}
\nomenclature{$\SS(\Ac)$}{Stabile Hülle}
\index{Stabile Hülle}
  Sei $\Ac$ eine Teilmenge von $\Ms^2$. Der durch $\Ac$ erzeugte stabile
  Unterraum \emph{$\SS(\Ac)$} ist der Durchschnitt aller $\Ac$ enthaltenden
  abgeschlossenen stabilen Unterräume.\fish
\end{definition}

\begin{rem*}
Wenn $\Ac$ bereits stabil ist, so folgt $\Ac = \SS(\Ac)$.\map
\end{rem*}

\subsection{Schwache und starke Orthogonalität}

Wir wollen nun die Geometrie des Hilbertraumes $\Ms^2$ genauer untersuchen.

\begin{definition}
\label{defn:5.4}
\index{orthogonal!schwach}
\index{orthogonal!stark}
Zwei Martingale $N,M \in \Ms^2$ heißen
\begin{defnenum}
\item \emph{schwach orthogonal}, falls $\lin{N,M}_{\Hs^2} = \E(N_\infty
M_\infty)=0$, und
\item \emph{stark orthogonal}, falls $L=NM$ ein gleichgradig
integrierbares Martingal ist.~\fish  
\end{defnenum}
\end{definition}

Es stehen uns also zwei Orthogonalitätsbegriffe zur Verfügung. Einerseits die
schwache Orthogonalität, welche durch das Innenprodukt
$\lin{\cdot,\cdot}_{\Hs^2}$ definiert wird, und andererseits die starke
Orthogonalität, definiert durch die Martingaleigenschaft des Produktes $NM$.
Sind zwei Martingale stark orthogonal, so sind sie auch schwach orthogonal, die
Umkehrung gilt jedoch im Allgemeinen nicht. Starke Orthogonalität ist somit
tatsächlich eine >>stärkere Forderung<< als schwache Orthogonalität.

\begin{definition}
\label{defn:5.5}
\nomenclature{$\Ac^\perp$}{Schwach orthogonales Komplement}
\nomenclature{$\Ac^\times$}{Stark orthogonales Komplement}
\index{orthogonale Mengen}
Sei $\Ac \subset \Ms^2$, so bezeichnen $\Ac^\perp$ und $\Ac^\times$
die zu $\Ac$ schwach bzw.\ stark orthogonalen Mengen.\fish
\end{definition}

Jede Teilmenge $\Ac\subset\Ms^2$ besitzt also zwei verschiedene orthogonale
Komplemente, einerseits das durch das Innenprodukt
definierte orthogonale Komplement $\Ac^\perp$, und andererseits 
$\Ac^\times$. Offenbar gilt $\Ac^\times \subset\Ac^\perp$, denn schwache
Orthogonalität ist weniger restriktiv als starke. Weiterhin ist $\Ac^\perp$
stets abgeschlossen -- für $\Ac^\times$ gilt mehr.

\begin{lemma}
\label{lem:5.1}
Sei $\Ac \subset \Ms^2$. Dann ist $\Ac^\times$ abgeschlossen und stabil.\fish
\end{lemma}
\begin{proof}
Wir zeigen zunächst, dass $\Ac^\times$ abgeschlossen ist. Sei dazu $M_n$ eine
Folge in $\Ac^\times$ mit $M_n\Mto M$, so ist zu zeigen, dass auch
$M\in\Ac^\times$. Sei also $N\in\Ac$ ein Martingal, dann gilt
\begin{align*}
MN = M_-\bullet N + N_-\bullet M + [M,N]. 
\end{align*}
Die ersten beiden Summanden sind als stochastische
Integrale bezüglich $L^2$"=Martingalen selbst wieder $L^2$"=Martingale nach
Lemma~\ref{lem:4.3}. Wenn auch $[M,N]$ ein Martingal ist, dann ebenso $MN$ und
die starke Orthogonalität wäre gezeigt. Die quadratische Kovariation ist eine
symmetrische, positiv semidefinite Bilinearform und erfüllt daher die
Cauchy-Schwarz-Ungleichung. Es gilt also
\begin{align*}
\abs{[M^n,N]-[M,N]}  =
\abs{[M^n-M,N]} \le [M^n-M,M^n-M][N,N].
\end{align*}
Somit folgt $[N,M^n]_t\lto{1} [N,M]_t$ für jedes $t\ge 0$, denn
\begin{align*}
\norm{[M^n,N]_t-[M,M]_t}_{L^1}
&\le \norm{[M^n-M,M^n-M]_t[N,N]_t}_{L^1}\\ 
&\le \norm{M_t^n-M_t}_{L^2}\norm{N_t}_{L^2}\\
&\le \norm{M^n-M}_{\Ms^2}\norm{N}_{\Ms^2}\to 0.
\end{align*}
Ferner ist $M^nN$ für jedes $n\ge 1$ ein Martingal, also auch $[M^n,N]$, d.h.
\begin{align*}
[N,M^n]_s = \E ([N,M^n]_t\mid \Fc_s),\qquad s\le t.
\end{align*}
Die linke Seite konvergiert in $L^1$ gegen $[N,M]_s$, während die rechte
Seite in $L^1$ gegen $\E([N,M]_t\mid\Fc_s)$ konvergiert, denn die bedingte
Erwartung ist stabil unter $L^1$-Konvergenz. Also ist $[N,M]$ ein
$L^1$-Martingal und $\Ac^\times$ ist abgeschlossen.

Sei nun $M\in\Ac^\times$ und $N\in\Ac$, dann ist $MN$ ein Martingal. Sei weiter
$T$ eine Stoppzeit, dann ist auch $(MN)^T$ ein Martingal nach dem Optional
Stopping Theorem. Somit ist $\Ac^\times$ abgeschlossen unter Stoppen und
folglich stabil.\qed
\end{proof}

Als nächstes erarbeiten wir die folgende Charakterisierung der
starken Orthogonalität, die wir noch häufig verwenden werden.

\begin{lemma}
\label{lem:5.2}
Seien $N,M \in \Ms^2$. Dann sind die folgenden Behauptungen äquivalent:
\begin{equivenum}
\item $M$ und $N$ sind stark orthogonal,
\item $\SS(M)$ und $N$ sind stark orthogonal,
\item $\SS(M)$ und $\SS(N)$ sind stark orthogonal,
\item $\SS(M)$ und $N$ sind schwach orthogonal,
\item $\SS(M)$ und $\SS(N)$ sind schwach orthogonal.\fish
\end{equivenum}
\end{lemma}
\begin{proof}
(i)$\Rightarrow$(ii): Seien $M$ und $N$ stark orthogonal, dann ist
$M\in\setd{N}^\times$ und folglich gilt $\SS(M)\subset \setd{N}^\times$, denn
$\setd{N}^\times$ ist abgeschlossen, stabil und enthält $M$.

(ii)$\Rightarrow$(iii): Seien $\SS(M)$ und $N$ stark orthogonal, dann ist
$N\in \SS(M)^\times$, und folglich gilt $\SS(N)\subset \SS(M)^\times$.

(iii)$\Rightarrow$(v): Klar nach Definition.

(v)$\Rightarrow$(iv): Ebenfalls klar.

(iv)$\Rightarrow$(i): Seien $\SS(M)$ und $\SS(N)$ schwach orthogonal. Um die
starke Orthogonalität von $M$ und $N$ zu zeigen, genügt es zu zeigen, dass
$[M,N]$ ein Martingal darstellt. Wir verwenden dazu Satz~\ref{prop:1.17}, und
zeigen, dass für jede beschränkte Stoppzeit $T$ gilt $\E[M,N]_T = 0 =
\E[M,N]_0$, denn $M_0 = N_0 = 0$. Offenbar gilt
\begin{align*}
\E[M,N]_T = \E[M^T,N]_\infty = 0,
\end{align*}
denn $M^T\in\SS(M)$, und $\SS(M)$ und $N$ sind schwach orthogonal.\qed
\end{proof}

\begin{rem*}
Der letzte Beweisschritt (iv)$\Rightarrow$(i) wäre nicht möglich gewesen, wenn
$M$ und $N$ lediglich als schwach orthogonal vorausgesetzt würden.\map 
\end{rem*}

\subsection{Eine Folgerungen}

Mit Hilfe dieser Charakterisierung erhalten wir eine erste Antwort auf unsere
ursprüngliche Frage, wann für zwei Martingal $L$, $M$ eine Darstellung
\begin{align*}
L = H\bullet M
\end{align*}
mit $H$ geeignet existiert.

\begin{theorem}
\label{prop:5.2}
Seien $M^1,\ldots,M^n \in \Ms^2$ und $M^i$ und $M^j$ stark orthogonal für
$i\neq j$. Dann besteht $\SS(M^1,\ldots,M^n)$ aus der Menge aller stochastischen
Integrale
\begin{align*}
\sum_{1\le i\le n} H^i \bullet M^i
\end{align*}
mit vorhersagbaren Prozessen $H^i$, für die gilt, dass
\begin{align*}
\E \int_0^\infty (H^i_s)^2 \, \ddd[M^i,M^i]_s < \infty.\fish
\end{align*}
\end{theorem}

\begin{proof}
Wir definieren die Menge der Prozesse mit der gewünschten Darstellung
\begin{align*}
\Ic \defl \setdef*{\sum_{1\le i\le n} H^i\bullet
M^i}{H^i\in\Pc\text{ und }\E \int_0^\infty (H_s^i)^2\, \ddd[M^i,M^i]_s <
\infty},
\end{align*}
und zeigen, dass $\Ic = \SS(M^1,\ldots,M^n)$. Zunächst gilt $\Ic \subset
\SS(M^1,\ldots,M^n)$ nach Satz~\ref{prop:5.1}. Zeigen wir also, dass $\Ic$
abgeschlossen und stabil ist, so folgt $\Ic = \SS(M^1,\ldots,M^n)$.

Um zu zeigen, dass $\Ic$ abgeschlossen ist, setzen wir
\begin{align*}
L^2(M^j) \defl \setdef*{H\in\Pc}{\norm{H}_{L^2(M^j)}^2 \defl \E\int_0^\infty
H_s^2 \,\ddd[M^j,M^j]_s < \infty},
\end{align*}
und betrachten die lineare Abbildung
\begin{align*}
A: L^2(M^1)\oplus \cdots\oplus L^2(M^n) \to \Ms^2,\quad (H^1,\ldots,H^n)\mapsto
\sum_{1\le i\le n} H^i\bullet M^i.
\end{align*}
Wir berechnen nun direkt, dass
\begin{align*}
\norm{A(H^1,\ldots,H^n)}_{\Hs^2} &= \E\left[\sum_{1\le i\le n} H^i\bullet
M^i,\sum_{1\le j\le n} H^j\bullet M^j\right]\\ & = 
\E\sum_{1\le i,j\le n} H_iH_j\bullet [M^i,M^j]\\
&= \sum_{1\le i\le n} \E\, H_i^2\bullet [M^i,M^i]\\
&= \sum_{1\le i\le n}\norm{H^i}_{L^2(M^i)},
\end{align*} 
also ist $A$ eine Isometrie. Die direkte Summe $L^2(M^1)\oplus \cdots\oplus
L^2(M^n)$ ist vollständig und ihr Bild unter $A$ ist gerade $\Ic$, folglich ist
auch $\Ic$ vollständig und insbesondere abgeschlossen.

Ferner ist $\Ic$ abgeschlossen unter Stoppen, denn für $H^j\bullet M^j\in \Ic$
und eine Stoppzeit $T$ gilt
\begin{align*}
(H^j\bullet M^j)^T = (H^j\Id_{[0,T]})\bullet M^j\in\Ic,
\end{align*}
also ist $\Ic$ stabil.\qed  
\end{proof}


\begin{theorem}
\label{prop:5.3}
Sei $\Ac$ eine stabile Teilmenge von $\Ms^2$. Dann ist auch $\Ac^\perp$ stabil
und jedes Martingal $M\in \Ac^\perp$ ist stark orthogonal zu $\Ac$, also
\begin{align*}
\Ac^\perp = \Ac^\times \quad \text{und} \quad \SS(\Ac)=\Ac^{
\perp\perp} =
\Ac^{\times\perp} = \Ac^{\times\times}.\fish
\end{align*}
\end{theorem}
\begin{proof}
Wir zeigen zuerst, dass $\Ac^\perp = \Ac^\times$. Sei dazu $M\in\Ac$ und
$N\in\Ac^\bot$. Da $\Ac$ stabil ist, gilt $\SS(M)\subset \Ac$, also ist $N$
schwach orthogonal zu $\SS(M)$. Nach Lemma~\ref{lem:5.2} ist daher $N$ stark
orthogonal zu $M$, und da $M\in\Ac$ beliebig war, folgt dass $N\in\Ac^\times$.
Also ist $\Ac^{\perp} \subset \Ac^\times$ und folglich gilt $\Ac^\perp =
\Ac^\times$.

Ferner ist $\Ac^\perp=\Ac^\times$ stabil nach Lemma~\ref{lem:5.1}, also erhalten
mit exakt derselben Argumentation, dass
\begin{align*}
(\Ac^\times)^\perp  = (\Ac^\perp)^\perp = (\Ac^\perp)^\times =
(\Ac^\times)^\times.
\end{align*}
Da $\Ms^2$ ein Hilbertraum ist, gilt außerdem $\bar{\Ac} = \Ac^{\perp\perp}$.
Nach Voraussetzung ist $\Ac$ aber stabil und damit abgeschlossen, folglich gilt
$\SS(\Ac) = \Ac = \bar{\Ac}$.\qed
\end{proof}


Aus den Hilbertraumeigenschaften von $\Ms^2$ folgt, dass wenn $\Ac$ ein
abgeschlossener \textit{Teilraum} von $\Ms^2$ ist, zu jedem Martingal
$M\in\Ms^2$ eine \textit{eindeutige} Zerlegung
\begin{align*}
M = A + B,\qquad A\in\Ac,\quad B\in\Ac^\perp
\end{align*}
existiert. Wenn $\Ac$ nun einen stabilen Unterraum darstellt, dann ist $\Ac$
insbesondere abgeschlossen und es gilt $\Ac^\perp = \Ac^\times$, also erhalten
wir dieselbe Zerlegung für $\Ac^\times$.

\begin{korollar}
\label{cor:5.1}
Sei $\Ac$ ein stabiler Unterraum von $\Ms^2$. Dann besitzt jedes $M\in
\Ms^2$ eine eindeutige Zerlegung $M=A+B$ mit $A\in\Ac$ und
$B\in\Ac^\times$.\fish
\end{korollar}

Als direkte Anwendung erhalten wir einen weiteren Darstellungssatz.

\begin{korollar}
\label{cor:5.2}
  Seien $N,M\in\Ms^2$ und $L$ die Projektion von $N$ auf $\SS(M)$. Dann
  existiert ein vorhersagbarer Prozess $H$ mit $L=H\bullet M$.
\end{korollar}
\begin{proof}
Setzen wir $\Ac\defl \SS(M)$, so ist $\Ac$ stabil und folglich existiert ein
eindeutig bestimmtes $L\in\SS(M)$, so dass $N = L + C$, wobei $C\in\Ac^\times$.
Nach Satz~\ref{prop:5.2} lässt sich jedes Element in $\SS(M)$ als stochastisches
Integral bezüglich $M$ darstellen, also insbesondere $L$.\qed
\end{proof}

\section{Martingaldarstellung}

Unser Ziel ist es nun, Bedingungen dafür anzugeben, dass zu gegebenen
Martingalen $M^1,\ldots,M^n$, >>möglichst alle<< Prozesse über eine
Darstellung wie in Satz \ref{prop:5.2} angegeben verfügen. Wir präzisieren dies
mit folgender Definition.

\begin{definition}
\index{vorhersagbare Darstellungseigenschaft}
Sei $\Ac$ eine endliche Menge schwach orthogonaler Martingale in
$\Ms^2$. Dann besitzt $\Ac$ die \emph{vorhersagbare Darstellungseigenschaft},
falls
\begin{align*}
\Ms^2 =
\setdef*{\sum_{1\le i\le n} H^i \bullet M^i}{M^i\in\Ac,\; H^i\in\Pc \text{
 mit } \E \int_0^\infty (H^i_s)^2 \, \ddd[M^i,M^i]_s < \infty}.\fish
\end{align*}
\end{definition}

Somit können wir ein hinreichendes Kriterium dafür formulieren, dass sich die
Menge der darstellbaren Prozesse nicht weiter vergrößern lässt.

\begin{korollar}
\label{cor:5.3}
Sei $\Ac = \setd{ M^1,\ldots,M^n } \subset \Ms^2$ mit stark orthogonalen $M^i$
und $M^j$ für $i\neq j$. Ferner gelte für jedes $N \in \Ms^2$, dass $N \perp
\Ac$ (im starken Sinne) bereits $N=0$ impliziert. Dann besitzt $\Ac$ die
vorhersagbare Darstellungseigenschaft.\fish
\end{korollar}
\begin{proof}
Aus Satz~\ref{prop:5.2} folgt, dass alle Prozesse aus $\SS(\Ac)$
über die gewünschte Darstellung verfügen. Weiterhin erhalten wir mit Korollar
\ref{cor:5.2} die orthogonale Zerlegung
\begin{align*}
\Ms^2 = \SS(\Ac)\oplus \SS(\Ac)^\times.
\end{align*}
Wählen wir nun ein $N\in\SS(\Ac)^\times$, so gilt auch
$N\in\Ac^\times$. Aus den Voraussetzungen folgt somit $N= 0$, also ist $\Ms^2 =
\SS(A)$, und dies war zu zeigen.\qed
\end{proof}

Im verbleibenden Teil dieses Kapitels suchen wir nach Wegen, die Bedingungen an
die Menge $\Ac$ aus obigem Korollar weiter abzuschwächen.

\subsection{$\Ms^2$-Martingalmaße} 

\begin{definition}
\nomenclature[S]{$\m^2(\Ac)$}{$\Ms^2$-Martingalmaße zu $\Ac$}
Sei $\Ac \subset \Ms^2$. Die Menge \emph{$\m^2(\Ac)$} der
\emph{$\Ms^2$-Martingalmaße zu $\Ac$} ist die Menge aller
Wahrscheinlichkeitsmaße $Q$ auf $\Fc=\sigma(\F)$, für die gilt:
\begin{defnenum}
\item $Q \ll P$,
\item $Q=P$ auf $\Fc_0$, und
\item jedes $X\in\Ac$ ist ein $L^2$-beschränktes $Q$-Martingal zur Filtration
$\F$.\fish
\end{defnenum}
\end{definition}

\begin{lemma}
Die Menge $\m^2(\Ac)$ ist konvex.\fish
\end{lemma}
\begin{proof}
Seien $Q$, $R$ zwei Maße in $\m^2(\Ac)$ und $0 < \lambda < 1$, so haben wir zu
zeigen, dass auch $S\defl \lambda Q + (1-\lambda)R\in\m^2(\Ac)$. Als
Konvexkombination von Wahrscheinlichkeitsmaßen ist auch $S$ ein solches.
Ferner lassen sich 1. und 2. direkt an der Definition von $S$ ablesen. Sei nun
$X\in\Ac$, dann ist $X$ sowohl ein $Q$-Martingal, als auch ein $R$-Martingal.
Sei weiter $A\in\Fc_s$ und $t\ge s$, dann gilt
\begin{align*}
\E_S(X_t \Id_A) &= \lambda \E_Q(X_t\Id_A) + (1-\lambda)\E_R(X_t\Id_A)\\
&= \lambda \E_Q(X_s\Id_A) + (1-\lambda)\E_R(X_s\Id_A)
= \E_S(X_s \Id_A),
\end{align*}
und folglich ist $X$ ein $S$-Martingal. Weiterhin verifizieren wir
\begin{align*}
\sup_{t\ge 0} \E_S X_t^2 \le
\lambda \sup_{t\ge 0} \E_Q X_t^2
+
(1-\lambda) \sup_{t\ge 0} \E_Q X_t^2
< \infty,
\end{align*}
denn $X$ ist $L^2$-beschränkt bezüglich $Q$ und $P$. Also gilt $S\in\m^2(\Ac)$
und die Konvexität ist gezeigt.\qed
\end{proof}

Betrachtet man lineare Abbildungen -- wie beispielsweise Integraloperatoren --
auf konvexen Mengen, so ist deren Verhalten maßgeblich durch das Verhalten
auf den extremalen Punkten der konvexen Menge bestimmt. In unserem Fall
spielen die Extremalpunkte der $\Ms^2$-Martingalmaße eine zentrale
Rolle.

\begin{definition}
\index{Extremalpunkt}
Das Maß $Q\in \m^2(\Ac)$ ist ein \emph{Extremalpunkt} von $\m^2(\Ac)$, falls
aus der Darstellung $Q= \lambda R + (1-\lambda) S$ mit $R,S\in \m^2(\Ac)$,
$R\neq S$ und $\lambda\in [0,1]$ entweder $\lambda=0$ oder $\lambda=1$
folgt.\fish
\end{definition}

Wenn die stabile Hülle einer Menge $\Ac$ bereits der gesamte Raum
$\Ms^2$ ist, so muss $P$ ein Extremalpunkt sein, was uns eine handhabbare
Charakterisierung von Extremalpunkten liefert.

\begin{theorem}
\label{prop:5.4}
Sei $\Ac\subset\Ms^2$. Falls $\SS(\Ac) = \Ms^2$, dann ist $P$ ein
Extremalpunkt von $\m^2(\Ac)$.\fish
\end{theorem}
\begin{proof}
Angenommen $P$ ist als Konvexkombination von $\m^2(\Ac)$-Maßen gegeben, also 
\begin{align*}
P = \lambda Q + (1-\lambda)R,\qquad 0 < \lambda < 1,\qquad Q,\; R\in\m^2(\Ac).
\end{align*}
So gilt $Q \le \lambda^{-1} P$, und $Q$ ist absolutstetig bezüglich $P$. Also
ist
\begin{align*}
L_t \defl \E_P\left(\frac{\dQ}{\dP}\mid \Fc_t \right),
\end{align*}
ein $P$-Martingal, und da außerdem $L_\infty = \dQ/\dP \le \lambda^{-1}$, ist
$L$ sogar beschränkt. Ferner gilt für jedes $A\in\Fc_0$, dass
\begin{align*}
\E_P (L_0\,\Id_A) = \E_P (L_\infty\,\Id_A) = \E_Q (\Id_A) = \E_P(\Id_A),
\end{align*}
denn $P=Q$ auf $\Fc_0$. Folglich ist $L_0\equiv 1$ \fs, und $L-L_0$ ist ein
Martingal in $\Ms^2$.

Sei nun $X\in\Ac$, dann ist $X$ sowohl ein $Q$-Martingal als auch ein
$R$-Martingal, und da $P = \lambda Q+ (1-\lambda)R$, ist $X$ somit auch ein
$P$-Martingal. Sei ferner $A\in\Fc_s$ für ein $s\le t$, dann gilt
\begin{align*}
\E_P(X_tL_t\Id_A) = \E_P(X_tL_\infty\Id_A) = \E_Q(X_t\Id_A) = \E_Q(X_s\Id_A) = 
\E_P(X_sL_s\Id_A),
\end{align*}
also ist $XL$ ein $P$-Martingal. Somit ist auch $XL - X = X(L-L_0)$ ein
$P$-Martingal, und folglich sind $X$ und $L-L_0$ stark orthogonal bezüglich $P$.
Daher ist $L-L_0$ nach Lemma~\ref{lem:5.1} sogar stark orthogonal zu
$\SS(\Ac) = \Ms^2$, d.h. $L-L_0 \in\SS(\Ac)^\times = (0)$ also ist $L\equiv 1$
und $P=Q$.\qed
\end{proof}

Der vorangegangene Satz besagt, dass, wenn jedes zu $\Ac$ stark orthogonale
$\Ms^2$-Martingal bereits identisch Null ist, $P$ ein Extremalpunkt ist. Der
folgende Satz liefert die Umkehrung allerdings nur für beschränkte Martingale.

\begin{theorem}
\label{prop:5.5}
Sei $\Ac\subset\Ms^2$. Ist $P$ ein Extremalpunkt von $\m^2(\Ac)$, dann ist
jedes beschränkte und in Null startende $P$-Martingal, welches stark orthogonal
zu $\Ac$ ist, gleich Null.\fish
\end{theorem}
\begin{proof}
Sei $L$ ein beschränktes und zu $\Ac$ stark orthogonales $P$-Martingal mit $L_0
= 0$. Wählen wir ein $c > 0$ mit $\abs{L}\le c$ und definieren
\begin{align*}
\dQ \defl \left(1-\frac{L_\infty}{2c} \right)\dP,\qquad
\dR \defl \left(1+\frac{L_\infty}{2c} \right)\dP,
\end{align*}
so sind $Q$ und $R$ wohldefinierte Wahrscheinlichkeitsmaße, denn $\E_PL_\infty =
0$. Wir zeigen nun, dass $Q$ und $R$ Maße in $\m^2(\Ac)$ sind. Zunächst gilt $P
= (Q+R)/2$, und daher ist $Q\ll P$ und $R\ll P$. Weiterhin folgt mit $L_0 =
0$, dass $Q=R=P$ auf $\Fc_0$. Sei ferner $X\in\Ac$, so ist zu zeigen, dass $X$
bezüglich $Q$ ein $L^2$-beschränktes Martingal ist. Nach Konstruktion ist
$1-(2c)^{-1}L_\infty \le 3/2$, also folgt
\begin{align*}
\sup_{t\ge 0} \E_Q X_t^2 = 
\sup_{t\ge 0} \E_P \left(\left(1- \frac{L_\infty}{2c} \right)X_t\right)
\le \frac{3}{2} \sup_{t\ge 0} \E_P X_t^2 < \infty,
\end{align*}
und daher ist $X$ auch $L^2(Q)$-beschränkt. Ferner gilt für jedes
$A\in\Fc_s$ und $s\le t$, dass
\begin{align*}
\E_Q (X_t\, \Id_A)
&= \E_P \left(\left(1-\frac{L_t}{2c}\right)X_t \Id_A \right)\\
% &= \E_P(X_t\Id_a) - \frac{1}{2c}\E_P(L_t X_t \Id_A)\\
% &= \E_P(X_s\Id_a) - \frac{1}{2c}\E_P(L_s X_s \Id_A)\\
&= \E_P \left(\left(1-\frac{L_s}{2c}\right)X_s \Id_A \right)=
\E_Q \left(X_s \Id_A \right),
\end{align*}
denn $X$ ist ein $P$-Martingal, und ebenso $XL$, denn nach Voraussetzung gilt ja
$L\perp \Ac$. Somit ist gezeigt, dass $Q\in\m^2(\Ac)$.

Aus Symmetriegründen folgt eine analoge Aussage für $R$. Also sind $Q$ und $R$
Maße in $\m^2(\Ac)$. Nach Voraussetzung ist $P = (Q+R)/2$ aber ein
Extremalpunkt, also gilt $Q=R=P$ und daher ist $L_\infty = 0$. Also ist
$L\equiv 0$, und dies war zu zeigen.\qed
\end{proof}

Wir wenden uns nun wieder unserem Ziel für dieses Kapitel zu, aus möglichst
schwachen Bedingungen an die Menge $\Ac$ die vorhersagbare
Darstellungseigenschaft zu folgern. Ist $P$ ein Extremalpunkt von $\m^2(\Ac)$,
dann gilt nach vorigem Satz, dass $N\equiv 0$ für jedes \textit{beschränkte}
Martingal $N\in\Ac^\times$ folgt. Um Korollar \ref{cor:5.3} anwenden
zu können, und so die vorhersagbare Darstellungseigenschaft zu folgern,
benötigen wir letztere Eigenschaft aber für beliebige \textit{eventuell
unbeschränkte} Martingale $N\in\Ac^\times$. Im folgenden Satz sichern
wir eine minimale Regularität für alle Martingale in $\Ac^\times$, und erhalten so,
dass all diese Martingale \textit{lokal beschränkt} sind. Wir können dann wie
üblich verfahren.

\begin{theorem}
\label{prop:5.6}
Sei $\Ac = \setd{M^1,\ldots,M^n} \subset\Ms^2$, wobei $M^i$ stetig und $M^i$
und $M^j$ stark orthogonal für $i\neq j$. Ist $P$ ein Extremalpunkt von
$\m^2(\Ac)$, dann besitzt $\Ac$ die vorhersagbare Darstellungseigenschaft.\fish
\end{theorem}
\begin{proof}
Aus den Voraussetzungen des Satzes folgt, dass jedes beschränkte
$P$-Martingal bereits stetig ist -- für Details
siehe \cite[Theorem 40, p. 186]{Protter:2004wfa}.

Wir zeigen nun, dass dies dann sogar für jedes gleichgradig integrierbare
$P$-Martingal gilt. Sei also $N$ gleichgradig integrierbar, so 
besitzt $N$ einen Abschluss $N_\infty$.
Definieren wir nun
\begin{align*}
N_t^n \defl \E_P(N_\infty\Id_{[\abs{N_\infty} \le n]}\mid \Fc_t),
\end{align*}
so ist $N^n$ für jedes $n\ge 1$ ein beschränktes Martingal und daher stetig.
Weiterhin ist $\abs{N-N^n}$ für jedes $n\ge 1$ ein positives Submartingal und
mit der Maximalungleichung von Doob \ref{prop:1.19} folgt für jedes $\ep > 0$
\begin{align*}
P\left[ \sup_{t\ge 0} \abs{N_t-N_t^n}\ge \ep\right]
&\le \frac{1}{\ep} \E_P\abs{N_\infty^n - N_\infty}\\
&= 
\frac{1}{\ep} \E_P(\abs{N_\infty}\Id_{[\abs{N_\infty} \le n]}) \to 0,\qquad n\to
\infty,
\end{align*}
denn der Integrand verschwindet und $\abs{N_\infty}$ ist eine integrierbare
Majorante. Somit gilt $(N-N^n)^* \Pto 0$, und es existiert eine fast sicher
konvergente Teilfolge. Also konvergiert $N^{n_k}\unito M$ für fast alle Pfade
gleichmäßig auf ganz $\R_+$. Als gleichmäßige Limites stetiger Pfade sind
daher fast alle Pfade von $M$ stetig.

Sei nun $N\in\Ac^\times$, dann ist $N$ nach dem eben gezeigten stetig und
folglich lokal beschränkt. Es existiert also eine Fundamentalfolge $(T^m)$ von
Stoppzeiten, so dass $N^{T^m}$ beschränkt ist. Weiterhin folgt aus der
Stabilität, dass $N^{T^m}\in\Ac^\times$. Somit können wir Satz \ref{prop:5.5}
anwenden und erhalten $N^{T^m}\equiv 0$. Da dies für jedes $m\ge 1$ gilt, folgt
$N\equiv 0$. Somit ist Korollar \ref{cor:5.3} anwendbar und die vorhersagbare
Darstellungseigenschaft von $\Ac$ folgt.\qed
\end{proof}

\begin{prop*}[Quintessenz]
Sei $\Ac=\setd{M^1,\ldots,M^n}$ eine Menge von stetigen und paarweise
stark orthogonalen $P$-Martingalen. Es lässt sich jedes $P$-Martingal
genau dann als eine Summe von Integralen $H^i\bullet M^i$ mit vorhersagbaren
Prozesse $H^i$ darstellen, falls $P$ ein extremaler Punkt in der Menge
derjenigen Wahrscheinlichkeitsmaße ist, bezüglich derer die $M^i$ auch
Martingale sind.\fish
\end{prop*}

\subsection{Martingaldarstellung bezüglich einer Brownschen Bewegung}

Jetzt behandeln wir den wichtigen Spezialfall, dass $\Ac$ aus den Komponenten
einer $n$-dimensionalen Standard-Brownschen Bewegung besteht.

\begin{theorem}
\label{rep-thm}
Sei $X=(X^1,\ldots,X^n)$ eine $n$-dimensionale Brownsche Bewegung und $\F$ die
dazugehörige vollständige natürliche Filtration. Dann besitzt jedes lokal
quadratintegrierbare lokale Martingal $M$ bzgl.\ $\F$ die Darstellung
\begin{align*}
M_t=M_0+ \sum_{i=1}^n \int_0^t H_s^i \,\ddd X_s^i
\end{align*}
mit vorhersagbaren Prozessen $H^i\in L(X^i)$.\fish
\end{theorem}
\begin{proof}
Sei $t_0 > 0$, dann ist $Y\defl X^{t_0}=(Y^1,\ldots,Y^n)$ ein $L^2$-beschränktes
Martingal, und folglich
\begin{align*}
\Ac \defl \setd{Y^1,\ldots,Y^n}\subset\Ms^2.
\end{align*}
Sei $Q\in\m^2(\Ac)$, dann ist $Y$ auch bezüglich $Q$ ein Martingal, und folglich
nach dem Satz von L\'{e}vy \ref{prop:2.26} eine Brownsche"=Bewegung, denn
$[Y^i,Y^j]_t = t\delta_{ij}$ ist unabhängig vom zugrundeliegenden
Wahrscheinlichkeitsmaß.
Für jedes $t\ge s$ ist daher $Y_{t}-Y_s$ multivariat Standard-normalverteilt
sowohl bezüglich $P$ als auch bezüglich $Q$. Somit sind die Bildmaße identisch,
also
\begin{align*}
P_{Y_t-Y_s} = Q_{Y_t-Y_s},\qquad t\ge s,
\end{align*}
und folglich gilt $P = Q$ auf $\Fc_{t_0}$, das heißt $\m^2(\Ac) =\setd{P}$.
Daher ist $P$ trivialerweise eine Extremalpunkt und $\Ac$ besitzt nach Satz
\ref{prop:5.6} die vorhersagbare Darstellungseigenschaft.

Sei nun $M$ ein lokal quadratintegirerbares lokales Martingal, so existiert auf
$[0,t_0]$ eine Fundamentalfolge von Stoppzeiten $T^m$, so dass
$M^{T^m}\in\Ms^2$. Aus der vorhersagbaren Darstellungseigenschaft von $\Ac$
folgt, daher dass vorhersagbare Prozesse $H^{i,m,0}$ existieren mit
\begin{align*}
M^{T_m} = \sum_{1\le i\le n} H^{i,m,0}\bullet Y^i,\qquad m\ge 1.
\end{align*}
Zusammengenommen erhalten wir auf $[0,t_0]$ eine gemeinsame Darstellung 
\begin{align*}
M = \sum_{1\le i\le n} H^{i,0}\bullet Y^i,\qquad H^{i,0} \defl \sum_{m\ge 1}
H^{i,m,0}\Id_{(T^{m-1},T^m]},
\end{align*}
wobei $H^{i,0}\in L(Y^i)$ für $1\le i\le n$.

Wir wählen nun eine deterministische Fundamentalfolge $t_1 < t_2 < \dotsm
\uparrow \infty$ und verfahren wie zuvor mit $t_l$ anstatt $t_0$. So erhalten
wir eine Folge von Prozessen $H^{i,l}$, so dass auf $[0,t_l]$
\begin{align*}
M = \sum_{1\le i\le n} H^{i,l}\bullet (X^i)^{t_l},\qquad l\ge 1.
\end{align*}
Setzen wir nun
\begin{align*}
H^i \defl \sum_{l\ge 1} H^{i,l}\Id_{(t_{i-1},t_i]},\qquad 1\le i\le n. 
\end{align*}
so folgt für $M$ auf ganz $\R_+$ die gewünschte Darstellung
\begin{align*}
M = \sum_{1\le i\le n} H^i\bullet X^i.\qed
\end{align*}
\end{proof}

Die Stetigkeit der Brownschen Bewegung schlägt sich signifikant in der durch sie
erzeugten natürlichen Filtration nieder. Letztere lässt nämlich keine unstetigen
Martingale zu.

\begin{korollar}
\label{cor:5.4}
Unter den Voraussetzungen von Satz~\ref{rep-thm} ist jedes lokale
Martingal $M$ bzgl.\ $\F$ stetig.\fish
\end{korollar}
\begin{proof}
Nach Satz~\ref{rep-thm} ist $P$ ein Extremalpunkt von $\m^2(\Ac)$. Aus dem
Beweis von Satz~\ref{prop:5.6} folgt weiterhin, dass jedes gleichgradig
integrierbare $P$-Martingal bereits stetig ist. Stoppen ergibt die
Behauptung.\qed
\end{proof}

Im letzten Schritt zeigen wir, dass sich jedes lokale Martingal als
stochastisches Integral bezüglich einer Brownschen Bewegung schrieben lässt.

\begin{korollar}
Unter den Voraussetzungen von Satz~\ref{rep-thm} besitzt jedes lokale Martingal
$M$ bzgl.\ $\F$ die Darstellung
\begin{align*}
M_t=M_0+ \sum_{i=1}^n \int_0^t H_s^i \,\dX_s^i
\end{align*}
mit vorhersagbaren Prozessen $H^i\in L(X^i)$.\fish
\end{korollar}
\begin{proof}
Nach Korollar~\ref{cor:5.4} ist jedes lokale Martingal unter diesen
Voraussetzungen stetig und daher insbesondere lokal quadratintegrierbar. Somit ist
Satz~\ref{rep-thm} anwendbar und die Behauptung folgt.\qed
\end{proof}


