\chapter{Zerlegung von Semimartingalen}

\section{Klassische Semimartingale}

Wir haben Semimartingale als gute Integratoren eingeführt, und bisher
ausschließlich mit dieser Sichtweise gearbeitet. Am Ende von Abschnitt 2-B wurde
jedoch bereits kurz angedeutet, dass man Semimartingale auch anders
motivieren kann, nämlich als diejenigen Prozesse, die sich in eine Summe aus
einem lokalen Martingal und einem FV-Prozess zerlegen lassen. In der Literatur
spricht man in diesem Fall auch von >>klassischen Semimartingalen<<.
Es stellt sich aber heraus dass klassische Semimartingale und Semimartingale ein
und dasselbe sind. Jedes Semimartingal lässt sich wie beschrieben
zerlegen, und umgekehrt ist durch jede solche Zerlegung ein Semimartingal
gegeben.

Wir wollen diese Zerlegungseigenschaft nun genauer untersuchen und einige
Folgerungen ableiten. Den Beweis, dass jedes Semimartingal eine solche Zerlegung
zulässt, überspringen wir allerdings und verweisen stattdessen auf die Literatur
-- beispielsweise Protter \cite{Protter:2004wfa}.

\begin{definition}
\index{Semimartingal!klassisches}
\label{defn:3.1}
Ein adaptierter \cadlag\ Prozess $X$ ist ein
\emph{klassisches Semimartingal}, falls es ein lokales Martingal $M$ und einen
Prozess $A$ von beschränkter Variation auf kompakten Mengen gibt, so dass
$M_0=A_0=0$ und 
\begin{align*}
X_t=X_0+M_t+A_t.\fish
\end{align*}
\end{definition}

%Die Sätze \ref{semi}, \ref{doob-meyer}, \ref{FundiLokMart} und
%\ref{specialsemi} werden wir nicht beweisen.

Jedes klassische Semimartingal ist ein Semimartingal -- eine etwas schwächere
Version dieser Aussage haben wir mit Satz \ref{prop:2.6} gezeigt. Die Umkehrung
ist ebenfalls wahr, deren Beweis erfordert allerdings erheblich mehr Aufwand.

\begin{theorem}
\label{semi}
\label{prop:3.1}
 Jedes Semimartingal ist ein klassisches
  Semimartingal und umgekehrt.
\end{theorem}

Eine unmittelbare Konsequenz dieses Satzes ist, dass jedes lokale
Martingal mit \cadlag Pfaden ein Semimartingal ist. Bisher konnten wir dies
lediglich für stetige oder zumindest lokale
quadratintegrierbare lokale Martingale zeigen.

\begin{korollar}
\label{cor:3.1}
 Jedes lokale Martingal mit \cadlag\ Pfaden ist ein
  Semimartingal.
\end{korollar}

\begin{definition}
\index{Prozess!vorhersagbarer}
\index{Prozess!predictable}
\index{Prozess!previsible}
\label{defn:3.2}
Die vorhersagbare $\sigma$-Algebra $\Pc$ ist die kleinste
  $\sigma$-Algebra auf $\R_+\times \Omega$, die alle Prozesse aus $\L$
  messbar macht. Die Klasse der $\Pc$-messbaren Prozesse wird ebenfalls mit
  $\Pc$ bezeichnet. Ein Prozess aus $\Pc$ wird \emph{vorhersagbarer Prozess
  (predictable, previsible)} genannt.\fish
\end{definition}

Wir werden noch sehen, dass die Klasse der vorhersagbaren Prozesse
tatsächlich größer als $\L$ ist.

Im Allgemeinen ist die Zerlegung, die man erhält, wenn man ein Semimartingal als
klassisches Semimartingal auffasst, nicht eindeutig. Durch zusätzliche
Einschränkungen an die Zerlegung kann man jedoch Eindeutigkeit erzwingen, indem
man beispielsweise die Vorhersagbarkeit des FV-Anteils fordert.

\begin{theorem}[Doob-Meyer-Zerlegung eines Submartingals]
\label{doob-meyer}
\label{prop:3.2}
Ist $X$ ein lokales Submartingal, dann existiert genau ein lokales Martingal $M$
und genau ein wachsender vorhersagbarer lokal integrierbarer Prozess $A$, so
dass $M_0=A_0=0$ und
\begin{align*}
X_t=X_0+M_t+A_t.\fish
\end{align*}
\end{theorem}

Jedes stetige lokale Martingal ist von unbeschränkter Totalvariation. Umgekehrt
lässt sich vermuten, dass sich der irreguläre Teil eines unstetigen lokalen
Martingals in Form eines Prozesses von beschränkter Variation
abspalten lässt. Folgender Satz formalisiert diese Vermutung.

\begin{theorem}[Fundamentalsatz für lokale Martingale]
\label{FundiLokMart} 
\label{prop:3.3}
\index{Fundamentalsatz!für lokale Martingale}
Sei
  $M$ ein lokales Martingal und $\beta > 0$. Dann existieren zwei lokale
  Martingale $N$ und $D$, so dass $D$ ein Prozess mit beschränkter Variation
  auf kompakten Mengen ist, die Sprünge von $N$ durch $2\beta$ beschränkt sind
  und
\begin{align*}
M=N+D.\fish
\end{align*}
\end{theorem}

Vorhersagbarkeit ist eine angenehme Eigenschaft, über die sich Prozesse bis zu
einem gewissen Grad kontrollieren lassen. Nach der Doob-Meyer-Zerlegung lässt
sich jedes Submartingal eindeutig in ein lokales Martingal und einen
vorhersagbaren FV-Prozess zerlegen. Natürlich hat nicht jedes Semimartingal
diese spezielle Eigenschaft.

\begin{definition}
\index{Semimartingal!spezielles}
\label{defn:3.3}
Sei $X$ ein Semimartingal. Besitzt $X$ eine Zerlegung der Form
\begin{align*}
X_t=X_0+M_t+A_t
\end{align*} 
mit einem lokalen Martingal $M$, einem vorhersagbaren Prozess $A$ von
beschränkter Variation auf kompakten Mengen und $M_0=A_0=0$, dann wird $X$
\emph{spezielles Semimartingal} genannt.\fish
\end{definition}

Bei der Doob-Meyer-Zerlegung impliziert die Vorhersagbarkeit des FV-Anteils die
Eindeutigkeit der Zerlegung. Dies gilt allgemein für Semimartingale.

\begin{theorem}
\label{specialsemi}
\label{prop:3.4}
 Ist $X$ ein spezielles Semimartingal, dann
  ist die Zerlegung mit einem vorhersagbaren Prozess $A$ eindeutig. Die
  Zerlegung eines Semimartingals ist im allgemeinen jedoch nicht eindeutig.\fish
\end{theorem}

Wir sind nun in der Lage tatsächlich zu beweisen, dass die Klasse der lokalen
Martingale abgeschlossen unter stochastischer Integration ist.

\begin{theorem}
\label{prop:3.5}
Sei $M$ ein lokales Martingal und $H \in \L$. Dann ist das stochastische
Integral $H \bullet M$ wieder ein lokales Martingal.\fish
\end{theorem}
\begin{proof}
Sei $M$ ein lokales Martingal und $H\in \L$. So ist $M$ nach Korollar
\ref{cor:3.1} ein Semimartingal und $H\bullet M$ ist wohldefiniert. Nach dem
Satz von Doob-Meyer \ref{prop:3.3} existiert für $\beta > 0$ eine Zerlegung
\begin{align*}
M = N+A,
\end{align*}
in lokale Martingale $N$ und $A$, wobei $A$ ein FV-Prozess ist, und $\abs{\Delta
N} \le \beta$ gilt.
Insbesondere ist $N$ lokal beschränkt, und folglich lokal
quadratintegrierbar. Also ist $H\bullet N$ nach Satz \ref{prop:2.14} ein
lokales Martingal.

Wir müssen also nur noch zeigen, dass $H\bullet A$ ebenfalls
ein lokales Martingal ist. Sei dazu $\sigma^n = (T_i^n)$ eine Partition des
Intervalls $[0,\infty)$, welche gegen die Identität konvergiert. Dann gilt nach
Satz \ref{prop:2.15}, dass
\begin{align*}
\sum_{i} H_{T_i^n}(A^{T_{i+1}^n}-A^{T_i^n}) \ucpto H\bullet A.
\end{align*}
Wir fixieren ein $t > 0$ und wählen eine Teilfolge $n_k$, so dass
\begin{align*}
\sum_{i} H_{T_i^n}(A_t^{T_{i+1}^n}-A_t^{T_i^n}) \fsto (H\bullet A)_t.\tag{*}
\end{align*}
Durch Stoppen können wir erreichen, dass $H$ beschränkt ist. Weiterhin ist 
$A$ von beschränkter Variation, also können wir erneut zu einer Teilfolge
übergehen, um zu erreichen, dass (*) auch in $L^1$ konvergiert. Somit folgt
\begin{align*}
\E(H\bullet A_t \mid \Fc_s) = \lim\limits_{n\to \infty}
\sum_i 
\E (H_{T_i^n}(A_t^{T_{i+1}^n}-A_t^{T_i^n})\mid \Fc_s).
\end{align*}
Nun ist $H_{T_i^n}$ $\Fc_{T_i^n}$-messbar, und folglich gilt
\begin{align*}
\E (H_{T_i^n}(A_t^{T_{i+1}^n}-A_t^{T_i^n})\Id_{[T_i^n \le  s]}\mid \Fc_s)
&= 
H_{T_i^n}\Id_{[T_i^n \le  s]} \E (A_t^{T_{i+1}^n}-A_t^{T_i^n}\mid
\Fc_s)\\
&= 
H_{T_i^n}\Id_{[T_i^n \le  s]} (A_s^{T_{i+1}^n}-A_s^{T_i^n}),
\end{align*}
sowie
\begin{align*}
\E (H_{T_i^n}(A_t^{T_{i+1}^n}-A_t^{T_i^n})\Id_{[T_i^n >  s]}\mid \Fc_s)
&=
\E (\E(H_{T_i^n}(A_t^{T_{i+1}^n}-A_t^{T_i^n})\Id_{[T_i^n > 
s]}\mid\Fc_{T_i})\mid \Fc_s)\\
&= \E (H_{T_i^n}\Id_{[T_i^n > 
s]}\E(A_t^{T_{i+1}^n}-A_t^{T_i^n}\mid\Fc_{T_i})\mid \Fc_s)\\
&= 0.
\end{align*}
Damit haben wir gezeigt, dass
\begin{align*}
\E ((H\bullet A)_t\mid \Fc_s) = 
\lim\limits_{n\to \infty}
\sum_i 
H_{T_i^n}(A_s^{T_{i+1}^n}-A_s^{T_i^n})
= (H\bullet A)_s,
\end{align*}
also ist $H\bullet A$ ebenfalls ein lokales Martingal.\qed
\end{proof}

Zum Abschluss geben wir noch ein hinreichendes Kriterium dafür an, dass ein
Semimartingal speziell ist.

\begin{theorem}
\label{prop:3.6}
Ein klassisches Semimartingal mit beschränkten Sprüngen ist ein spezielles
Semimartingal.\fish
\end{theorem}

Jedes stetige Semimartingal ist also ein spezielles Semimartingal, also
insbesondere der Wiener-Prozess. Aber auch der Poisson-Prozess ist ein
spezielles Semimartingal, denn $\Delta N_t$ ist entweder Null oder Eins.

\section{Der Satz von Girsanov}

Die Zerlegung eines klassischen Semimartingals $X$ in
\begin{align*}
X = M+A
\end{align*}
mit einem lokalen Martingal $M$ und einem Prozess $A$ von beschränkter
Variation, hängt vom zugrundeliegenden Wahrscheinlichkeitsmaß $P$ ab, denn die
lokale Martingaleigenschaft wird durch $P$ definiert, und auch die
Semimartingaleigenschaft von $X$ ist $P$-abhängig. In vielen Anwendungen ist
jedoch eine ganze Familie von Wahrscheinlichkeitsmaßen gegeben. Es stellt
sich in natürlicher Weise die Frage, unter welchen Bedingungen auch
bezüglich eines anderen Wahrscheinlichkeitsmaßes $Q$ eine analoge Zerlegung
$X=N+B$ existiert.

\begin{defn*}
\index{Maß!absolutstetig}
\nomenclature{$Q\ll P$}{$Q$ ist absolutstetig bezüglich $P$}
Ein Maß $Q$ heißt \emph{absolutstetig} bezüglich dem Wahrscheinlichkeitsmaß $P$
auf $(\Omega,\Fc)$ (kurz $Q \ll P$), falls jede $P$-Nullmenge auch eine
$Q$-Nullmenge ist.\fish
\end{defn*}

Nach dem Satz von Radon-Nikodym besitzt jedes $P$-absolutstetige
Wahrscheinlichkeitsmaß $Q$ eine Radon-Nikodym-Ableitung $Z=\dQ/\dP$, so dass
$Q(A)=\int_A Z\, \dP$ für alle $A \in \Fc$.

\begin{theorem}
\label{prop:3.7}
Das Wahrscheinlichkeitsmaß $Q$ sei absolutstetig bezüglich
  $P$. Dann ist ein stochastischer Prozess $X$, der bezüglich $P$ ein
  Semimartingal ist, auch ein Semimartingal bezüglich $Q$ (kurz: ein
  $Q$-Semimartingal).\fish
\end{theorem}
\begin{proof}
Nach Definition ist $X$ genau dann ein totales $P$-Semimartingal, wenn die
Abbildung $I_X : \S_u \to L^0$ stetig bezüglich $P$ ist, d.h. wenn 
\begin{align*}
H_n\to H\text{ in }\S_u \quad \Rightarrow\quad H_n\bullet
X\Pto H\bullet X.
\end{align*}
Zeigen wir nun, dass aus $U_n\Pto U$ auch $U_n\overset{\mathbf Q}{\longto} U$
folgt, so ist $I_X$ auch $Q$-stetig, und folglich $X$ ein
$Q$-Semimartingal. Seien also $\ep > 0$ und $\delta > 0$, so gilt
\begin{align*}
Q([\abs{U_n-U}> \ep]) 
&\le \int_{[\abs{Z}\le k]}\Id_{[\abs{U_n-U}> \ep]}\abs{Z}\,\dP
+
\int_{[\abs{Z}> k]}\Id_{[\abs{U_n-U}> \ep]}\abs{Z}\,\dP\\
&\le k P([\abs{U_n-U}> \ep]) +
\int_{[\abs{Z}> k]}\abs{Z}\,\dP.
\end{align*}
Da $Z$ integrierbar ist, können wir $k$ so groß wählen, dass der zweite Summand
kleiner als $\delta$ wird. Ferner lässt sich aufgrund von $U_n\Pto U$ nun $n$ so
groß wählen, dass auch der erste Summand kleiner als $\delta$ wird. Schließlich
erhalten wir
\begin{align*}
Q([\abs{U_n-U}> \ep])  \to 0,\qquad n\to \infty.\qed
\end{align*} 
\end{proof}

\begin{definition}
\label{defn:3.4}
\index{Maß!äquivalent}
\nomenclature{$Q\sim P$}{$Q\ll P$ und $P\ll Q$}
 Zwei Wahrscheinlichkeitsmaße auf $(\Omega,\Fc)$ heißen
  \emph{äquivalent} ($P \sim Q$), falls $Q$ absolutstetig bezüglich $P$  und
  $P$ absolutstetig bezüglich $Q$ ist.\fish
\end{definition}

\begin{lem}
\label{lem:3.1}
Seien $Q \sim P$ und $Z_t=E_P\left( \frac{\dQ}{\dP} \mid \Fc_t \right)$. Genau
dann ist ein adaptierter \cadlag\ Prozess $M$ ein lokales Martingal bezüglich $Q$, falls $MZ$ ein lokales Martingal bezüglich
  $P$ ist.\fish
\end{lem}
\begin{proof}
Sei $Z_\infty = \frac{\dQ}{\dP}$ die Radon-Nikodym-Ableitung, so gilt
\begin{align*}
Z_t = \E_P(Z_\infty\mid \Fc_t) > 0\ \fs,
\end{align*}
und $Z$ ist ein $P$-Martingal. Weiterhin gilt
\begin{align*}
\E_P(M_tZ_t) = \E_P(M_t\E_P(Z_\infty\mid \Fc_t)) = \E_P(M_t Z_\infty) =
E_Q(M_t).
\end{align*}
Somit ist $MZ\in L^1(P)$ genau dann, wenn $M\in L^1(Q)$. Wir haben nun zu
zeigen, dass für alle $0 \le s \le t$,
\begin{align*}
\E_Q(M_t\mid \Fc_s) = M_s \iff 
\E_P(M_tZ_t\mid \Fc_s) = M_sZ_s.\tag{*}
\end{align*}
Dazu zeigen wir zunächst die verallgemeinerte Bayes-Formel
\begin{align*}
\E_Q(M_t\mid \Fc_s) = 
\frac{\E_P(M_tZ_t\mid \Fc_s)}{\E_P(Z_t\mid \Fc_s)}.\tag{**}
\end{align*}
Sei $A\in\Fc_s$, so gilt
\begin{align*}
%\E_Q(M_t \Id_A) =
%\E_Q(\E_Q(M_t\mid \Fc_s) \Id_A) =
\E_Q\left(\frac{\E_P(M_tZ_t\mid \Fc_s)}{\E_P(Z_t\mid \Fc_s)} \Id_A\right) &=
\E_P\left(\frac{\E_P(M_tZ_t\mid \Fc_s)}{Z_s} Z_s \Id_A\right) \\
&=  \E_P\left(M_t Z_t \Id_A\right)\\
&= \E_Q(M_t\Id_A) =  \E_Q(\E_Q(M_t\mid\Fc_s)\Id_A),
\end{align*}
und (**) ist gezeigt. Sei wieder $A\in\Fc_s$, dann gilt mit (**)
\begin{align*}
\E_Q
\E_P(\E_P(M_tZ_t\mid\Fc_s)\Id_A) &= 
\E_P(\E_Q(M_t\mid\Fc_s)Z_s\Id_A) \\
&= \E_Q((M_t\mid\Fc_s) \Id_A),
\end{align*}
und die Äquivalenz in (*) folgt.\qed
\end{proof}


\begin{theorem}[Satz von Girsanov-Meyer]
\index{Satz!von Girsanov-Meyer}
\label{prop:3.8}
 Seien $P$ und $Q$ zwei äquivalente
  Wahrscheinlichkeitsmaße. Ist $X$ unter $P$ ein klassisches Semimartingal mit
  der Zerlegung $X=M+A$, dann ist $X$ auch unter $Q$ ein klassisches
  Semimartingal mit der Zerlegung $X=L+C$, wobei
  \begin{align*}
  L_t = M_t - \int_0^t \frac{1}{Z_s}\, \ddd[Z,M]_s
  \end{align*}
ein lokales Martingal bezüglich $Q$ und $C=X-L$ von endlicher Variation
auf kompakten Mengen (bezüglich $Q$).\fish
\end{theorem}

\begin{rem*}
Der Integrand $Z_s^{-1}$ ist im Allgemeinen \textit{nicht} linksstetig. Dennoch
ist das Integral als Riemann-Stieltjes-Integral wohldefiniert, denn $[Z,M]$ ist
von beschränkter Variation auf kompakten Mengen.\map
\end{rem*}

\begin{proof}
Nach Satz \ref{prop:3.7} folgt, dass $X$ auch ein $Q$-Semimartingal ist. Wir
zeigen nun, dass durch $L$ ein lokales $Q$-Martingal definiert wird.
Aufgrund der Äquivalenz von $P$ und $Q$ ist $Z_\infty = \dQ/\dP$ wohldefiniert,
und
\begin{align*}
Z_t = \E_P(Z_\infty\mid\Fc_t),\qquad t \ge 0,
\end{align*}
ein $P$-Martingal. Umgekehrt ist $(1/Z_t)_{t\ge 0}$ ein $Q$-Martingal. Nun gilt 
\begin{align*}
\int_0^t \frac{1}{Z_s}\, \ddd[Z,M]_s
=
\int_0^t \frac{1}{Z_{s-}}\, \ddd[Z,M]_s
+
\sum_{0\le s \le t} \Delta \left(\frac{1}{Z_s}\right) \Delta [Z,M]_s.
\end{align*}
Mit der Formel für die partielle Integration folgt weiterhin
\begin{align*}
\int_0^t \frac{1}{Z_{s-}}\, \ddd[Z,M]_s
=
\frac{[Z,M]_t}{Z_t} -
\int_0^t [Z,M]_{s-}\,\ddd\left(\frac{1}{Z_{s-}}\right)
- \left[[Z,M],\frac{1}{Z}\right]_t.
\end{align*}
Schließlich gilt
\begin{align*}
\left[[Z,M],\frac{1}{Z}\right]_t &= \left[[Z,M],\frac{1}{Z^c}\right]_t + 
\sum_{0\le s \le t}\Delta \left(\frac{1}{Z_s}\right) \Delta [Z,M]_s, 
\end{align*}
und da $[Z,M]$ von beschränkter Variation ist, verschwindet der erste Term.
Zusammenfassend ergibt sich
\begin{align*}
L_t = M_t - \int_0^t \frac{1}{Z_s}\, \ddd[Z,M]_s = 
\frac{M_tZ_t - [Z,M]_t}{Z_t} +
\int_0^t [Z,M]_{s-}\,\ddd\left(\frac{1}{Z_{s}}\right).
\end{align*}
Nach Lemma \ref{lem:3.1} ist der erste Summand ein lokales $Q$-Martingal,
denn
\begin{align*}
Z_t\cdot \frac{M_tZ_t - [Z,M]_t}{Z_t} = 
M_tZ_t - [Z,M]_t = (M\bullet Z)_t + (Z\bullet M)_t,
\end{align*}
und sowohl $M$ als auch $Z$ sind lokale $P$-Martingale. Der zweite Summand
ist ebenfalls ein lokales $Q$-Martingal, denn $1/Z$ ist ein
$Q$-Martingal. Folglich ist $L$ ein lokales $Q$-Martingal. Schreiben wir nun
\begin{align*}
X = M+ A = \left(M - \int \frac{1}{Z}\,\ddd[Z,M]\right)
+
\left(\int \frac{1}{Z}\,\ddd[Z,M] + A\right)
\defr L + C,
\end{align*}
so ist $C$ als Integral bezüglich eines Integrators von beschränkter
Variation, selbst wieder von beschränkter Variation, und folglich $X = L+C$ die
gesuchte Zerlegung von $X$ bezüglich $Q$.\qed
\end{proof}

\begin{ex}
Es ist ein zentrales Problem der Finanzmathematik, zu einer gegebenen Zerlegung
$X = M+A$ ein Wahrscheinlichkeitsmaß $Q$ zu finden, so dass $X$ ein
lokales $Q$-Martingal ist. In diesem Fall nennt man $Q$ \emph{risikoneutrales
Maß} für $X$. Anschaulich gesprochen versucht man seine Sichtweise durch
Neubewertung der Wahrscheinlichkeit einzelner Ereignisse so zu verändern, dass
$X$ ein lokales Martingal wird.

Wir wollen diese Fragestellung nun genauer untersuchen.
Sei $(\Omega,\Fc,P;\F)$ ein gefilterter Wahrscheinlichkeitsraum, welcher die
üblichen Bedingungen bezüglich $\F$ erfüllt. Ferner sei ein Preisprozess
\begin{align*}
S = M+A
\end{align*}
gegeben, wobei $M$ ein lokales Martingal bezüglich $P$ darstelle. Wir suchen nun
ein zu $P$ äquivalentes Maß $Q$, so dass $S$ ein lokales $Q$-Martingal
darstellt.

Um die Fragestellung etwas zu vereinfachen, nehmen wir an, dass $S$
eine Lösung der folgenden stochastischen Differentialgleichung ist
\begin{align*}
\dS_s = h(s,S_s)\,\dB_s + b(s,S_s)\,\ds. 
\end{align*} 
Man nennt $h$ auch \emph{Volatilitätsfunktion} und $b$ \emph{Trendfunktion}.
Eine mögliche Wahl ist $h(s,S_s) = \sigma S_s$ und $g(s,S_s) = \mu S_s$, so wird
$S$ zu einem Black-Scholes-Preisprozess.

Sei nun $Q$ ein zu $P$ äquivalentes Wahrscheinlichkeitsmaß, welches wir später
geeignet wählen. Nach dem Satz von Radon-Nikodym existiert
$Z_\infty = \dQ/\dP$, und durch $Z_t = \E_P(Z_\infty\mid\Fc_t)$ ist ein
$P$-Martingal gegeben, welches $Q$ eindeutig bestimmt.
 In Kapitel 5 werden wir zeigen, dass zu
jedem lokalen Martingal $Z$ mit $\E Z = 1$ ein vorhersagbarer Prozess $J\in\Pc$
existiert, so dass
\begin{align*}
Z_t = 1 + \int_0^t J_s\,\dB_s.
\end{align*}
Unter schwachen Voraussetzungen an $J$ und $Z$ ist dann der Quotient 
\begin{align*}
H_s \defl \frac{J_s}{Z_s}
\end{align*}
wohldefiniert. Es gilt dann
\begin{align*}
Z_t = 1 + \int_0^t H_s Z_s\,\dB_s,
\end{align*}
und mit der Definition
\begin{align*}
N_t \defl \int_0^t H_s\,\dB_s
\end{align*}
erhalten wir die Integralgleichung
\begin{align*}
Z_t = 1 + \int_0^t Z_s\,\dN_s.
\end{align*}
Somit ist $Z = \Ep(N) = \Ep(H\bullet B)$ das stochastische Exponential von $N$.
Unser Ziel ist es nun, $H$ so zu bestimmen, dass $S$ bezüglich dem durch $Z$
induzierten Wahrscheinlichkeitsmaß $Q$ ein lokales Martingal darstellt. 
Gemäß dem Satz von Girsanovnov-Meyer \ref{prop:3.8} ist
\begin{align*}
L_t = \int_0^t h(s,S_s)\, \dB_s - \int_0^t \frac{1}{Z_s}
\,\ddd[Z,\int_0^\cdot h(r,S_r)\dB_r]_s
\end{align*}
ein lokales $Q$-Martingal. Ferner gilt $Z=1 + Z\bullet N$, also ist
\begin{align*}
[Z,\int_0^\cdot
h(r,S_r)\,\dB_r]_t = 
\int_0^t h(r,S_r)Z_r \,\ddd [B,N]_r
=
\int_0^t h(r,S_r)Z_rH_r \,\dr,
\end{align*}
denn $[B,N] = [B,H\bullet B] = H\bullet s$. Somit gilt
\begin{align*}
L_t &= \int_0^t h(s,S_s)\, \dB_s - \int_0^t \frac{1}{Z_s}h(s,S_s)Z_s H_s\,\ds\\
&= \int_0^t h(s,S_s)\, \dB_s - \int_0^t h(s,S_s) H_s\,\ds\\
&= \int_0^t h(s,S_s)\, (\dB_s - H_s\,\ds).
\end{align*}
Um zu erreichen, dass
\begin{align*}
L_t \overset{!}{=} S_t &= \int_0^t h(s,S_s)\,\dB_s + \int_0^t b(s,S_s)\,\ds\\
&= \int_0^t h(s,S_s)\, \left(\dB_s + \frac{b(s,S_s)}{h(s,S_s)}\ds\right),
\end{align*}
wählen wir also
\begin{align*}
H_s = -\frac{b(s,S_s)}{h(s,S_s)}.
\end{align*}
So wird $S$ ein lokales $Q$-Martingal. Setzen wir
\begin{align*}
M_t = B_t + \int_0^t \frac{b(s,S_s)}{h(s,S_s)}\,\ds,
\end{align*}
so verifiziert man, dass auch $M$ ein lokales $Q$-Martingal ist. Außerdem
gilt $[M,M]_t = [B,B]_t = t$, also ist $M$ nach dem Satz von Levy eine
Standard-Brownsche Bewegung bezüglich $Q$, und es gilt
\begin{align*}
S_t = \int_0^t h(s,S_s)\, \ddd M_s.
\end{align*}
Wir können $S$ also als Lösung folgender stochastischer Differentialgleichung
interpretieren
\begin{align*}
\dS_t = h(t,S_t)\,\ddd M_t.
\end{align*}

Zusammenfassend ist der Preisprozess $S$ genau dann ein Martingal bezüglich
eines gegebenen Wahrscheinlichkeitsmaßes $Q\sim P$, wenn das $P$-Martingal
\begin{align*}
Z_t = \E_P(Z_\infty \mid \Fc_t),\qquad Z_\infty = \frac{\dQ}{\dP}
\end{align*}
als stochastisches Exponential $Z_t = \Ep(N)_t$ gegeben ist. Hierbei ist
$N=H\bullet B$ gemäß obiger Rechnung festgelegt. 

In der ursprünglichen Fragestellung, welche durch die Finanzmathematik motiviert
wird, ist das Maß $Q$ jedoch nicht bekannt. Die Idee ist nun, dieses über den
Prozess $Z$ zu definieren, welcher als stochastisches Exponential von $N$
unabhängig von $Q$ wohldefiniert ist. Können wir zeigen, dass $Z$ durch diese
Definition ein abschließbares $P$-Martingal ist, so lässt sich $Q$ definieren
durch
\begin{align*}
Q(A) = \int_A Z_\infty\,\dP.
\end{align*}
Aus der Martingaleigenschaft folgt außerdem $\E Z_t = \E Z_0 = 1$, denn $Z_0
=1$. Folglich ist $Q$ automatisch ein Wahrscheinlichkeitsmaß. Als Integral
bezüglich einer Brownschen Bewegung ist $N$ ein lokales Martingal. Somit ist
auch $Z = \Ep(N)$ ein lokales Martingal. Wir benötigen also hinreichende
Kriterien dafür, dass ein lokales Martingal ein Martingal ist.\boxc
\end{ex}

Zur Konstruktion des äquivalenten Maßes $Q$ muss häufig entschieden werden, ob
das stochastische Exponential eines stetigen lokalen $P$-Martingals sogar ein
$P$-Martingal ist. Hierzu können die Kriterien von Kazamaki und Novikov
herangezogen werden.

\begin{lem}
\label{lem:3.2}
Sei $M$ ein stetiges lokales Martingal mit $M_0=0$. Dann ist $\Ep(M)$ ein
Supermartingal und 
\begin{align*}
\E \left(\Ep(M)_t \right) \le 1 ,\qquad t\ge 0.
\end{align*}
Falls $\E(\Ep(M)_t)=1$ für alle $t \ge 0$, so ist $\Ep(M)$ sogar ein
Martingal.\fish
\end{lem}
\begin{proof}
Jedes nichtnegative lokale Martingal $X$ ist ein Supermartingal. Um dies
einzusehen, betrachte eine $X$ lokalisierende Folge von Stoppzeiten $(T_n)$.
Fixieren wir ein $t\ge 0$, so folgt mit dem Lemma von Fatou und der
Martingaleigenschaft von $X^{T_n}$, dass
\begin{align*}
0\le \E X_t = \E \lim\limits_{n\to \infty} X_{t}^{T_n}
\le \liminf_{n\to\infty} \E X_t^{T_n} = \E X_0.
\end{align*}
Also ist $X_t$ integrierbar, und es folgt analog mit der bedingten Version des
Lemmas von Fatou für $s\le t$
\begin{align*}
\E(X_t\mid \Fc_s) = \E(\lim\limits_{n\to \infty} X_t^{T_n}\mid \Fc_s)
\le
\liminf\limits_{n\to \infty} 
\E( X_t^{T_n}\mid \Fc_s)
=
\liminf\limits_{n\to \infty} 
X_s^{T_n} = X_s.
\end{align*}
Ist der Erwartungswert von $X$ außerdem konstant, so ist $X$ ein Martingal, denn
aufgrund der Supermartingaleigenschaft ist $0 \le X_s - \E(X_t \mid \Fc_s)$ und
weiter
\begin{align*}
0 \le \E (X_s - \E(X_t \mid \Fc_s)) = \E X_s - \E X_t = 0.
\end{align*}
Nun ist $M$ ein lokales Martingal, also auch $X=\Ep(M)$, und mit dem bisher
gezeigten folgt die Behauptung.\qed
\end{proof} 

Zu verifizieren, dass $\E(\Ep(M)_t)=1$ für alle $t\ge 0$, ist in den Anwendungen
meist eine große Herausforderung. Wir erarbeiten daher nun einige, eventuell
einfacher zu verifizierende Kriterien,  welche $\E(\Ep(M)_t)\equiv 1$
sicherstellen. Zunächst aber zwei Ungleichungen.

\begin{theorem}
\label{prop:3.9}
\nomenclature[S]{$\bST$}{beschränkte Stoppzeiten}
Sei $M$ ein stetiges lokales Martingal und $T$ eine beschränkte Stoppzeit --
kurz $T\in\bST$. Dann gilt
\begin{align*}
\E \e^{\frac{1}{2}M_T} \le
\left(\E \e^{\frac{1}{2}[M,M]_T} \right)^{1/2}.\fish
\end{align*}
\end{theorem}
\begin{proof}
Nach Satz \ref{prop:2.24} ist das stochastische Exponential von $M$ gegeben
durch
\begin{align*}
\Ep(M)_t = \e^{M_t - \frac{1}{2}[M,M]_t} = \e^{M_t}\e^{- \frac{1}{2}[M,M]_t}. 
\end{align*}
Für eine beschränkte Stoppzeit $T$ gilt folglich
\begin{align*}
\e^{\frac{1}{2}M_T} = \Ep(M)_T^{1/2}\left(\e^{\frac{1}{2}[M,M]_T}\right)^{1/2}. 
\end{align*}
Eine Anwendung der Hölderungleichung zusammen mit $\E \Ep(M)_T \le 1$ ergibt
schließlich
\begin{align*}
\E \e^{\frac{1}{2}M_T} \le
(\E \Ep(M)_T)^{1/2}\left(\E \e^{\frac{1}{2}[M,M]_T}\right)^{1/2}
\le \left(\E \e^{\frac{1}{2}[M,M]_T}\right)^{1/2}.\qed
\end{align*}
\end{proof}


\begin{lemma}
\label{lem:3.3}
Seien $M$ ein stetiges lokales Martingal, $1 < p < \infty$ und
$\frac{1}{p} + \frac{1}{q} = 1$.  Falls
\begin{align*}
\sup\limits_{T\in\bST} \E\, \e^{\frac{\sqrt{p}}{2\sqrt{p}-1} M_T} < \infty, 
\end{align*}
dann ist $\Ep(M)$ ein $L^q$-beschränktes Martingal.\fish
\end{lemma}
\begin{proof}
Seien $p$ und $q$ konjugiert mit $1 < p <\infty$, sowie
\begin{align*}
r \defl \frac{\sqrt{p}-1}{\sqrt{p}+1},\qquad
s \defl \frac{\sqrt{p}+1}{2},
\end{align*}
dann sind auch $r$ und $s$ konjugiert und es gilt
\begin{align*}
\left(q - \sqrt{\frac{q}{r}} \right)s = \frac{\sqrt{p}}{2(\sqrt{p}-1)}
\end{align*}
Weiterhin gilt
\begin{align*}
\Ep(M)^q = \e^{qM-\frac{q}{2}[M,M]} = 
\e^{\sqrt{\frac{q}{r}}M - \frac{q}{2}[M,M]}
\e^{\left(q-\sqrt{\frac{q}{r}}\right)M}
\end{align*}
Sei nun $T$ eine beschränkte Zufallsvariable, dann folgt mit der Hölderschen
Ungleichung
\begin{align*}
\E \Ep(M)_T^q &\le 
\left(\E\, \e^{\sqrt{qr}M_T - \frac{qr}{2}[M,M]_T}\right)^{1/r}
\left(\E\, \e^{s\left(q-\sqrt{\frac{q}{r}}\right)M_T}\right)^{1/s}\\
&\le 
\left(\E\, \Ep\left(\sqrt{qr}M\right)_T\right)^{1/r}
\left(\E\, \e^{\frac{\sqrt{p}}{2(\sqrt{p}-1)}M_T}\right)^{1/s} \le K< \infty,
\end{align*}
wobei $K$ unabhängig von $T$ ist, denn der erste Faktor ist $\le 1$ nach Lemma
\ref{lem:3.2} und der zweite ist nach Voraussetzung beschränkt. Anwendung von
Satz \ref{prop:1.17} ergibt, dass $\Ep(M)^q$ ein $L^1$-beschränktes Martingal
ist, also ist $\Ep(M)$ ein $L^q$-beschränktes Martingal.\qed
\end{proof}

Mit Hilfe dieser Ungleichungen können wir nun ein handlicheres hinreichendes
Kriterium dafür beweisen, dass ein lokales Martingal ein Martingal ist.

\begin{theorem}[Kazamaki-Kriterium]
\index{Kazamaki-Kriterium}
\label{prop:3.10}
Seien $M$ ein stetiges lokales Martingal und
\begin{align*}
\sup\limits_{T\in \bST} \E\, \e^{\frac{1}{2}M_T} < \infty,
\end{align*}
dann ist $\Ep(M)$ ein gleichgradig integrierbares Martingal.\fish
\end{theorem}
\begin{proof}
Sei $0 < a < 1$ und $p > 1$ mit $\sqrt{p}(\sqrt{p}-1)^{-1} < a^{-1}$, so folgt
\begin{align*}
\sup_{T\in\bST} \E\, \e^{\frac{\sqrt{p}}{2(\sqrt{p}-1)}aM_T}
\le
\sup_{T\in\bST} \E\, \e^{\frac{1}{2}M_T} < \infty.  
\end{align*}
Also ist $\Ep(aM)$ nach Lemma \ref{lem:3.3} ein $L^q$-beschränktes Martingal,
wenn $q$ und $p$ konjugiert sind. Da $p>1$, ist auch $q>1$, und folglich ist
$\Ep(aM)$ gleichgradig integrierbar. Somit existiert ein Abschluss
$\Ep(aM)_\infty$ und $\E\, \Ep(aM)_\infty = \E\, \Ep(aM)_0 = 1$. 
Außerdem ist nach Lemma \ref{lem:3.2} $\Ep(M)$ ein nichtnegatives
Supermatringal mit $\E \Ep(M)_t \le 1$ für alle $t \ge 0$, also existiert
$\Ep(M)_\infty$ \fs.
Darüber hinaus
gilt
\begin{align*}
\Ep(aM) = \e^{aM-\frac{a^2}{2}[M,M]} = \e^{a^2M-\frac{a^2}{2}[M,M]}\e^{a(1-a)M}
= \Ep(M)^{a^2}\e^{a(1-a)M},
\end{align*}
also existiert auch $\e^{a(1-a)M_\infty} = \lim_{t\to\infty}
\e^{a(1-a)M}_t$ \fs.
Wenden wir die Hölderungleichung mit $\tilde p = a^{-2}$ und $\tilde q =
(1-a^{2})^{-1}$ an, erhalten wir
\begin{align*}
\E\, \Ep(aM)_\infty \le
(\E\,\Ep(M_\infty))^{a^2}
\left(\E\,\e^{\frac{a}{1+a}M_\infty}\right)^{1-a^2}. 
\end{align*}
Weiterhin ist
\begin{align*}
1-a^2 = 2a(1-a) \frac{1+a}{2a},
\end{align*}
wobei $\frac{1+a}{2a} > 1$. Da $x\mapsto x^{\frac{1+a}{2a}}$ konvex ist, folgt
mit der Jensenschen Ungleichung
\begin{align*}
\left(\E\,\e^{\frac{a}{1+a}M_\infty}\right)^{1-a^2}
\le 
\left(\E\,\e^{\frac{1}{2}M_\infty}\right)^{2a(1-a)}.
\end{align*}
Zusammenfassend gilt also
\begin{align*}
1 = \E\, \Ep(aM)_\infty \le 
(\E\,\Ep(M)_\infty)^{a^2}
\left(\E\,\e^{\frac{1}{2}M_\infty}\right)^{2a(1-a)} \to\quad
\E\,\Ep(M)_\infty,\qquad a \to 1,
\end{align*}
daher ist $1\le \E\, \Ep(M_\infty)\le 1$, und
folglich ist $\Ep(M)$ ein Martingal nach Lemma \ref{lem:3.3} mit $\E
\Ep(M)_t \equiv 1$.
Es verbleibt zu zeigen, dass $\Ep(M)$ gleichgradig integrierbar ist. Zunächst gilt aufgrund der
Martingaleigenschaft und der Nichtnegativität, dass
\begin{align*}
0 \le \Ep(M)_s - \E(\Ep(M)_t\mid \Fc_s),\qquad s\le t.
\end{align*}
Also gilt auch
\begin{align*}
0\le \Ep(M)_s - \E(\liminf_{t\to \infty}\Ep(M)_t\mid \Fc_s)
= 
\Ep(M)_s - \E(\Ep(M)_\infty\mid \Fc_s).
\end{align*}
Andererseits ist
\begin{align*}
0 \le \E\Ep(M)_s - \E\E(\Ep(M)_\infty\mid \Fc_s) = 1-1
= 0,
\end{align*}
und folglich $\Ep(M)_s =\E(\Ep(M)_\infty\mid \Fc_s)$. Also ist $\Ep(M)$ ein
abschließbares Martingal und nach Satz \ref{prop:1.11} gleichgradig
integrierbar.\qed
\end{proof}

Die folgende Bedingung ist zwar einschränkender, häufig aber einfacher zu
überprüfen.

\begin{theorem}[Novikov-Kriterium]
\index{Novikov-Kriterium}
\label{prop:3.11}
Sei $M$ ein stetiges lokales Martingal und
\begin{align*}
\E\, \e^{\frac{1}{2}[M,M]_\infty} < \infty,
\end{align*}
dann ist $\Ep(M)$ ein gleichgradig integrierbares Martingal.\fish
\end{theorem}
\begin{proof}
Wir müssen lediglich zeigen, dass das Kazamaki-Kriterium \ref{prop:3.10} erfüllt
ist. Sei also $T$ eine beschränkte Stoppzeit, so folgt mit Satz \ref{prop:3.9}
und der Monotonie des quadratischen Variationsprozesses, dass
\begin{align*}
\left(\E \e^{\frac{1}{2}M_T}\right)^2 \le
\E \e^{\frac{1}{2}[M,M]_T} 
\le
\E \e^{\frac{1}{2}[M,M]_\infty}.
\end{align*}
Die rechte Seite ist nach Voraussetzung endlich und zwar unabhängig von $T$,
somit ist das Kazamaki-Kriterium erfüllt.\qed
\end{proof}

\begin{ex*}[Fortsetzung von Beispiel 5]
Unser Ziel ist es zu zeigen, dass  
das stochastische Exponential
$Z = \Ep(N)$ ein $P$-Martingal ist, wobei $N=H\bullet B$. Wir betrachten dazu
\begin{align*}
[N,N]_\infty = \int_0^\infty H_s^2 \,\ds = \int_0^\infty
\left(\frac{b(s,S_s)}{h(s,S_s)}\right)^2\ds.
\end{align*}
Sofern also $H_s$ in $s$ quadratisch integrierbar ist, folgt
\begin{align*}
\E \e^{\frac{1}{2}[N,N]_\infty} < \infty,
\end{align*}
und das Novikov-Kriterium ist erfüllt.

Betrachten wir $S_t$ auf einem endlichen Zeitintervall $[0,T]$, so ist
\begin{align*}
\abs*{\frac{b(s,S_s)}{h(s,S_s)}} \le K < \infty
\end{align*}
eine grobe aber hinreichende Bedingung dafür, dass $Z$ ein $P$-Martingal ist.
Für den Black-Scholes-Preisprozess mit $b=\mu S$ und $h = \sigma S$ ist dies
offenbar der Fall.\boxc
\end{ex*}

Der folgende Satz betrifft speziell Brownsche Bewegungen $X$ mit Drift.

\begin{theorem}[Satz von Girsanov]
\index{Satz!von Girsanov}
\label{prop:3.12}
Seien $B$ eine Standard-Brownsche Bewegung, $H\in\L$ ein beschränkter
Prozess und
\begin{align*}
X_t\defl \int_0^t H_s \, \ds + B_t.
\end{align*}
Sei $T>0$ fest. Das Wahrscheinlichkeitsmaß $Q$ sei definiert durch
\begin{align*}
\frac{\dQ}{\dP} =
\exp \left( - \int_0^T H_s\,\dB_s - \frac{1}{2} \int_0^T H_s^2\,\ds\right).
\end{align*}
Dann ist $X=(X_t)_{t\in[0,T]}$ unter $Q$ eine Standard-Brownsche Bewegung.\fish
\end{theorem}
\begin{proof}
Wir zeigen zuerst, dass das Maß $Q$ wohldefiniert ist. Sei also $T > 0$ fest und
\begin{align*}
Z_T \defl \exp\left(-\int_0^T H_s\,\dB_s - \frac{1}{2} \int_0^T H_s^2\,\ds
\right).
\end{align*}
Setzen wir $N_t \defl -\int_0^t H_s\,\dB_s$, so ist $N$ ein stetiges lokales
Martingal bezüglich $P$. Außerdem gilt
\begin{align*}
[N,N]_T = \int_0^T H_s^2 \,\ds < \infty,
\end{align*}
denn $H$ ist beschränkt. Also ist $\Ep(N)$ ein gleichgradig integrierbares
$P$-Martingal nach dem Novikov-Kriterium \ref{prop:3.11}, und nach der Formel
für das stochastische Exponential \ref{prop:2.24} gilt
\begin{align*}
\Ep(N)_t = \exp\left(-\int_0^t H_s\,\dB_s - \frac{1}{2} \int_0^t H_s^2\,\ds
\right) = \E_P(Z_T\mid\Fc_t)\defr Z_t,\qquad 0\le t\le T.
\end{align*}
Ferner gilt $\E_P Z_T = \E_P Z_0 = 1$.
Somit durch $\dQ/\dP = Z_T$ ein wohldefiniertes
Wahrscheinlichkeitsmaß $Q$ gegeben.

% Als stochastisches Exponential löst $Z$ weiterhin die stochastische
% Integralgleichung
% \begin{align*}
% Z_t = 1 - \int_0^t Z_{s-}H_s \,\dB_s =  1 - \int_0^t Z_{s-}\,\dN_s.
% \end{align*}

Wir betrachten nun den Prozess
\begin{align*}
X_t = \int_0^t H_s\,\ds + B_t.
\end{align*}
Der erste Summand ist von beschränkter Variation, und $B$ ist ein $P$-Martingal.
Also ist der Satz von Girsanov-Meyer \ref{prop:3.8} anwendbar, der besagt, dass
$X = L + C$, wobei das lokale $Q$-Martingal $L$ gegeben ist durch
\begin{align*}
L_t = B_t - \int_0^t \frac{1}{Z_s}\,\ddd[Z,B]_s.
\end{align*}
Als stochastisches Exponential erfüllt $Z$ die Integralgleichung
\begin{align*}
Z_t = 1 - \int_0^t Z_{s-}H_s \,\dB_s =  1 - \int_0^t Z_{s-}\,\dN_s.
\end{align*}
Somit erhalten wir für das Differential des Kovariationsprozesses
\begin{align*}
\ddd[Z,B] = -\ddd[(ZH)\bullet B,B] = ZH\,\ds,
\end{align*}
und $L$ nimmt folgende Form an,
\begin{align*}
L_t = B_t - \int_0^t \frac{1}{Z_s}Z_s H_s\,\ds = 
B_t - \int_0^t H_s\,\ds = X_t.
\end{align*}
Also ist $X$ tatsächlich ein lokales $Q$-Martingal. 
Schreiben wir $A = -\int_0^t H_s\,\ds$, so folgt
\begin{align*}
[X,X]_t = [B+A,B+A]_t = [B,B]_t = t,
\end{align*}
denn $A$ ist von beschränkter Variation und $B$ ist stetig.
Folglich ist $X$ eine Standard-Brownsche-Bewegung bezüglich $Q$ nach dem Satz
von Lévy \ref{prop:2.26}, und dies war zu zeigen.\qed
\end{proof}

\begin{rem*}
Wir haben $H$ als beschränkt vorausgesetzt, damit wir das Novikov-Kriterium
leicht erfüllen können. Dazu würde es auch schon genügen, dass
\begin{align*}
\int_0^T H_s^2 \,\ds < \infty.
\end{align*}
Die Voraussetzungen an $H$ können aber noch wesentlich weiter abgeschwächt
werden.\map
\end{rem*}

